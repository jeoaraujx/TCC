A análise de grandes volumes de publicações científicas apresenta desafios complexos, principalmente na organização e categorização de padrões temáticos. Em resposta a esse cenário, este estudo propõe o desenvolvimento de um pipeline que combina o BERTopic e o GPT-4 para a análise de publicações científicas na plataforma SIMCC. O BERTopic é empregado para a modelagem de tópicos através do uso de embeddings contextuais, da redução de dimensionalidade com UMAP e do agrupamento com HDBSCAN. Paralelamente, o GPT-4 é utilizado para enriquecer semanticamente os clusters de tópicos identificados, gerando rótulos descritivos e precisos que complementam a modelagem. A base de dados do projeto provém do SIMCC, uma plataforma da Secretaria Estadual de Ciência, Tecnologia e Inovação da Bahia que centraliza e organiza dados de produção acadêmica de profissionais vinculados a instituições de ensino e pesquisa do estado, integrando informações de diversas fontes como Currículos Lattes, Sucupira e OpenAlex. O sistema oferece funcionalidades para o gerenciamento do conhecimento acadêmico. A integração desse pipeline à base de dados do SIMCC visa facilitar a análise e a visualização das publicações por meio de um modelo de mapeamento visual, semelhante ao WizMap, que organiza os tópicos em clusters. Essa abordagem busca aprimorar a categorização temática, contribuindo para uma compreensão mais estruturada e detalhada do acervo científico disponível na plataforma.

\textbf{Palavras-chave}: Processamento de Linguagem Natural; Inteligência Artificial; Modelagem de Tópicos; BERTopic; GPT-4; SIMCC; Análise de Publicações Científicas.