O crescente e contínuo volume de publicações científicas representa um desafio significativo para a organização, exploração e descoberta de padrões temáticos, especialmente quando as abordagens de busca se limitam a palavras-chave. Para mitigar essa lacuna, este estudo propõe a validação técnica de uma arquitetura de mapeamento interativo de conhecimento. Esta abordagem busca transformar a maneira como pesquisadores exploram acervos científicos, revelando conexões semânticas e estruturas latentes de forma intuitiva. A proposta foi desenvolvida como uma Prova de Conceito (PoC) utilizando um subconjunto de dados do Observatório de Dados Públicos de Ciência e Tecnologia da Bahia, que centraliza dados de produção acadêmica. O artefato tecnológico consiste em um \textit{pipeline} computacional que integra a modelagem de tópicos via \textit{BERTopic} (utilizando \textit{embeddings} contextuais, \textit{UMAP} e \textit{HDBSCAN}) com o refinamento de rótulos por meio da \textit{MMR} para aumentar a interpretabilidade. Os resultados são integrados à ferramenta \textit{WizMap}, permitindo a exploração interativa da estrutura dos dados. A validação experimental indicou a geração de agrupamentos semanticamente coesos e a viabilidade técnica da ferramenta \textit{WizMap} na projeção espacial de domínios correlatos. Os resultados indicam que a técnica de \textit{embeddings} mostrou-se adequada para organizar acervos científicos, oferecendo uma camada complementar aos métodos de buscas lexicais tradicionais. \vspace{\baselineskip}

\noindent \textbf{Palavras-chave:} Processamento de Linguagem Natural; Inteligência Artificial; Modelagem de Tópicos; BERTopic; Visualização de Dados; WizMap; Observatório de C\&T da Bahia.