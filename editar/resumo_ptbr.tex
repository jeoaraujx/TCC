O crescente e contínuo volume de publicações científicas representa um desafio significativo para a organização, exploração e descoberta de padrões temáticos, especialmente quando as abordagens de busca se limitam a palavras-chave. Para mitigar essa lacuna e aprimorar a capacidade de análise do conhecimento, este estudo propõe o desenvolvimento de uma \textbf{solução de mapeamento interativo de conhecimento}. Esta abordagem visa transformar a maneira como pesquisadores exploram acervos científicos, revelando conexões semânticas e tópicos emergentes de forma intuitiva. A proposta será aplicada ao acervo do \textbf{Observatório de dados públicos de ciência e tecnologia da Bahia}, que centraliza dados de produção acadêmica de fontes como Currículos Lattes, Sucupira e OpenAlex. O artefato tecnológico consiste em um \textit{pipeline} computacional que integra a modelagem de tópicos via \textbf{\gls{bertopic}} (utilizando \textit{embeddings} contextuais, \gls{umap} e \gls{hdbscan}) com o refinamento de rótulos por meio da \textbf{\gls{mmr}} (\textit{Maximal Marginal Relevance}) para aumentar a interpretabilidade. Os resultados do mapeamento são então integrados à ferramenta de visualização interativa \textbf{\gls{wizmap}}, permitindo a exploração visual e dinâmica da estrutura do conhecimento. Esta metodologia busca oferecer uma compreensão mais estruturada e acessível do acervo científico do Observatório, facilitando a identificação de tendências e a navegação por áreas de pesquisa.
\vspace{\baselineskip} % Espaçamento

\noindent
\textbf{Palavras-chave:} Processamento de Linguagem Natural; Inteligência Artificial; Modelagem de Tópicos; BERTopic; Visualização de Dados; WizMap; Observatório de C\&T da Bahia.