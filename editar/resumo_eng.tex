The growing and continuous volume of scientific publications represents a significant challenge for the organization, exploration, and discovery of thematic patterns, especially when search approaches are limited to keywords. To mitigate this gap and enhance knowledge analysis capabilities, this study proposes the development of an \textbf{interactive knowledge mapping solution}. This approach aims to transform the way researchers explore scientific collections, revealing semantic connections and emerging topics in an intuitive manner. The proposal will be applied to the collection of the \textbf{Observatory of Public Data on Science and Technology of Bahia}, which centralizes academic production data from sources such as Lattes Curricula, Sucupira, and OpenAlex. The technological artifact consists of a computational \textit{pipeline} that integrates topic modeling through \textbf{\gls{bertopic}} (using contextual \textit{embeddings}, \gls{umap}, and \gls{hdbscan}) with label refinement via \textbf{\gls{mmr}} (\textit{Maximal Marginal Relevance}) to enhance interpretability. The mapping results are then integrated into the interactive visualization tool \textbf{\gls{wizmap}}, enabling dynamic and visual exploration of the knowledge structure. This methodology seeks to provide a more structured and accessible understanding of the Observatory’s scientific collection, facilitating the identification of trends and navigation across research areas.
\vspace{\baselineskip}

\noindent
\textbf{Keywords:} Natural Language Processing; Artificial Intelligence; Topic Modeling; BERTopic; Data Visualization; WizMap; Science and Technology Observatory of Bahia.