The growing and continuous volume of scientific publications represents a significant challenge for the organization, exploration, and discovery of thematic patterns, especially when search approaches are limited to keywords. To mitigate this gap, this study proposes the technical validation of an interactive knowledge mapping architecture. This approach seeks to transform the way researchers explore scientific collections by intuitively revealing semantic connections and latent structures. The proposal was developed as a Proof of Concept (PoC) using a data subset from the Observatory of Public Data on Science and Technology of Bahia, which centralizes academic production data. The technological artifact consists of a computational pipeline that integrates topic modeling via \textit{BERTopic} (using contextual embeddings, \textit{UMAP}, and \textit{HDBSCAN}) with label refinement via \textit{MMR} to enhance interpretability. The results are integrated into the \textit{WizMap} tool, enabling interactive exploration of the data structure. Experimental validation indicated the generation of semantically cohesive clusters and the technical feasibility of the \textit{WizMap} tool in the spatial projection of correlated domains. The results indicate that the embedding technique proved adequate for organizing scientific collections, offering a complementary layer to traditional lexical search methods. \vspace{\baselineskip}

\noindent \textbf{Keywords:} Natural Language Processing; Artificial Intelligence; Topic Modeling; BERTopic; Data Visualization; WizMap; Science and Technology Observatory of Bahia.