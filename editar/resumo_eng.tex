The growing and continuous volume of scientific publications represents a significant challenge for the organization, exploration, and discovery of thematic patterns, especially when search approaches are limited to keywords. To mitigate this gap and enhance knowledge analysis capabilities, this study proposes the development of an interactive knowledge mapping solution. This approach sought to transform the way researchers explore scientific collections, revealing semantic connections and emerging topics in an intuitive manner. The proposal was applied to the collection of the Observatory of Public Data on Science and Technology of Bahia, which centralizes academic production data from sources such as Lattes Curricula, Sucupira, and OpenAlex. The technological artifact consists of a computational \textit{pipeline} that integrates topic modeling via \textit{BERTopic} (using contextual \textit{embeddings}, \textit{UMAP}, and \textit{HDBSCAN}) with label refinement via \textit{MMR} to enhance interpretability. The results are integrated into the \textit{WizMap} tool, enabling the interactive exploration of the data structure. The experimental validation indicated the generation of semantically cohesive clusters and the viability of the WizMap tool in the spatial projection of correlated domains. The results indicate that the \textit{embedding} technique proved adequate for organizing scientific collections, offering a practical alternative to traditional statistical methods.
\vspace{\baselineskip}

\noindent
\textbf{Keywords:} Natural Language Processing; Artificial Intelligence; Topic Modeling; BERTopic; Data Visualization; WizMap; Science and Technology Observatory of Bahia.