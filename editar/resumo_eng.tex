The analysis of large volumes of scientific publications presents complex challenges, mainly in the organization and categorization of thematic patterns. In response to this scenario, this study proposes the development of a pipeline that combines BERTopic and GPT-4 for the analysis of scientific publications on the SIMCC platform. BERTopic is used for topic modeling through the use of contextual embeddings, dimensionality reduction with UMAP, and clustering with HDBSCAN. Simultaneously, GPT-4 is utilized to semantically enrich the identified topic clusters, generating descriptive and precise labels that complement the modeling. The project's database comes from SIMCC, a platform from the State Secretariat for Science, Technology, and Innovation of Bahia, which centralizes and organizes academic production data from professionals affiliated with teaching and research institutions in the state. The system integrates information from various sources such as Lattes Curricula, Sucupira, and OpenAlex. The integration of this pipeline into the SIMCC database aims to facilitate the analysis and visualization of publications through a visual mapping model, similar to WizMap, which organizes topics into clusters. This approach seeks to improve thematic categorization, contributing to a more structured and detailed understanding of the available scientific collection.

\textbf{Key-words}: Natural Language Processing; Artificial Intelligence; Topic Modeling; BERTopic; GPT-4; SIMCC; Scientific Publications Analysis.