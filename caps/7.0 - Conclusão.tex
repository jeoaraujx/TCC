\chapter[CONCLUSÃO]{CONCLUSÃO}
\label{chap:conclusao}

Este Estudo de Caso propôs-se a enfrentar o desafio da exploração de grandes acervos de publicações científicas, cujas abordagens tradicionais de busca lexical (palavras-chave) limitam a descoberta de conhecimento. O objetivo central foi projetar, desenvolver e avaliar um artefato computacional, seguindo os princípios da \gls{dsr}, capaz de transformar o acervo textual do \textbf{Observatório de dados públicos de ciência e tecnologia da Bahia} em um mapa de conhecimento interativo.

Para atingir este objetivo, foi desenvolvido um \textit{pipeline} computacional (Capítulo \ref{chap:projeto_desenvolvimento}) que integra técnicas modernas de \gls{pln}. O \textit{pipeline} executa um fluxo de quatro etapas: (1) pré-processamento de dados textuais (títulos); (2) modelagem de tópicos com \gls{bertopic}, utilizando \textit{embeddings} \texttt{Sentence-BERT}, redução de dimensionalidade \gls{umap} e \textit{clusterização} \gls{hdbscan}; (3) refinamento de rótulos com \gls{mmr} para reduzir a redundância semântica; e (4) exportação dos dados processados (coordenadas 2D e metadados) para a ferramenta de visualização \gls{wizmap}.

A avaliação do artefato (Capítulo \ref{chap:resultados}), realizada sobre um \textit{dataset} de teste (NPAI), demonstrou a viabilidade técnica da solução. A validação quantitativa (Seção \ref{sec:validacao_quantitativa}) indicou uma alta diversidade de tópicos (0.9214), sugerindo baixa sobreposição lexical entre os temas. O escore de coerência \gls{npmi} (-0.2095) foi negativo, um resultado metodologicamente discutido como sendo uma limitação da métrica \gls{npmi} quando aplicada a modelos semânticos (como o \gls{bertopic}) e a textos curtos (títulos), onde a co-ocorrência de palavras é estatisticamente baixa.

A validação qualitativa (Seção \ref{sec:validacao_qualitativa}) complementou esta análise, demonstrando que, apesar do escore \gls{npmi}, os tópicos gerados apresentaram coerência semântica interpretável (ex: \enquote{Dengue e Arboviroses}, \enquote{Robótica Educacional}). A visualização final no \gls{wizmap} (Figura \ref{fig:wizmap_geral}) demonstrou a capacidade do artefato em posicionar tópicos semanticamente próximos em regiões adjacentes do mapa, oferecendo uma interface navegável para a exploração da paisagem de pesquisa.

Considera-se, portanto, que os objetivos delineados pela metodologia \gls{dsr} foram atingidos no que tange à construção e validação de um protótipo funcional. O artefato representa uma solução viável para o problema de descoberta de conhecimento no acervo do Observatório.

\section[Trabalhos Futuros]{Trabalhos Futuros}
\label{sec:trabalhos_futuros}

As limitações identificadas durante o desenvolvimento (Seção \ref{sec:limitacoes}) e a natureza de protótipo deste estudo abrem caminho para diversas frentes de trabalho futuro:

\begin{itemize}
    \item \textbf{Validação em Larga Escala:} Aplicar o \textit{pipeline} sobre o acervo de dados completo e dinâmico do Observatório de dados públicos de ciência e tecnologia da Bahia, o que constitui o próximo passo natural para validar a solução em um ambiente de produção.

    \item \textbf{Ajuste Fino de Hiperparâmetros:} A execução em larga escala exigirá uma nova etapa de \textit{tuning} (ajuste fino) dos hiperparâmetros (ex: \texttt{n\_neighbors} no \gls{umap} e \texttt{min\_cluster\_size} no \gls{hdbscan}) para adequá-los a um volume de dados massivo.

    \item \textbf{Enriquecimento do Texto de Entrada:} Incluir os \textbf{resumos} (\textit{abstracts}) das publicações, além dos títulos, no processo de modelagem. Textos mais longos podem fornecer mais contexto, potencialmente gerando tópicos mais ricos e impactando positivamente as métricas de coerência estatística como o \gls{npmi}.

    \item \textbf{Integração do Artefato:} Evoluir o protótipo (atualmente executado no Google Colab) para um módulo de \textit{software} integrado à arquitetura de produção do Observatório (Seção \ref{ssec:base_observatorio}), permitindo a atualização automática e periódica do mapa de conhecimento.
\end{itemize}