\section{Síntese do Referencial Teórico}
\label{sec:consideracoes_finais_referencial}

Em síntese, este capítulo consolidou a fundamentação teórica necessária para o desenvolvimento da pesquisa. Inicialmente, discutiram-se as limitações dos métodos tradicionais de recuperação da informação, evidenciando a necessidade de abordagens baseadas em Processamento de Linguagem Natural, especificamente o uso de embeddings contextuais e da arquitetura de Transformadores. Na sequência, o detalhamento técnico do modelo BERTopic e da ferramenta WizMap forneceu os subsídios para a construção do artefato computacional proposto. Com o referencial estabelecido, o capítulo seguinte contextualiza este estudo em relação aos trabalhos correlatos e ao estado da arte.