\chapter[INTRODUÇÃO]{INTRODUÇÃO}

O cenário da pesquisa científica global tem testemunhado um crescimento exponencial da produção científica nas últimas décadas, resultando em um vasto volume de dados que desafia os métodos tradicionais de organização e análise desse acervo. Para navegar nessa imensidão de informações, pesquisadores confiam em plataformas de busca, como \textit{Web of Science}\footnote{Disponível em: \url{https://access.clarivate.com/login?app=wos}.}, \textit{Scopus}\footnote{Disponível em: \url{https://www.scopus.com/home.uri}.} e \textit{IEEE Xplore}\footnote{Disponível em: \url{https://ieeexplore.ieee.org/}.}, utilizando principalmente palavras-chave. Contudo, essa abordagem de recuperação de informações é limitada pela ambiguidade e pela diversidade do léxico científico, o que frequentemente resulta em buscas que não retornam a completude esperada e na dificuldade de identificar tendências emergentes na literatura \cite{Galli_2024}.

Segundo \citeonline{Datchanamoorthy_2023}, a complexidade inerente a esses acervos e a necessidade de uma análise mais profunda têm impulsionado a aplicação de técnicas avançadas de \gls{pln}. Esse avanço, que une ciência da informação, \gls{ia} e linguística computacional, posiciona essas áreas como fundamentais na construção de soluções para a gestão do conhecimento acadêmico. Estudos como os de \citeonline{Mohammadi_2020} e \citeonline{Xie_2020}, que analisaram tendências de pesquisa em \textit{Big Data} por meio de mineração de texto, destacam a relevância da integração de técnicas de modelagem de tópicos com modelos baseados em transformadores.

Essa arquitetura de modelos, os Transformadores, introduzida por \citeonline{vaswani_2017}, revolucionou o campo da \gls{pln} com seu mecanismo de autoatenção (\textit{self-attention}). Modelos subsequentes, como o \gls{bert}\footnote{Disponível em: \url{https://huggingface.co/docs/transformers/model_doc/bert}.}, proposto por \citeonline{Devlin_2019}, passaram a capturar relações contextuais em textos com alta eficiência. A partir dessa base, surgiram os \textit{embeddings}, representações numéricas que codificam o significado semântico de palavras e frases, superando as limitações de modelos tradicionais de \textit{bag-of-words} e de modelagem de tópicos como o \gls{lda} \cite{Galli_2024}.

Nesse contexto, a técnica de \gls{bertopic}\footnote{Disponível em: \url{https://github.com/MaartenGr/BERTopic}.}, proposto por \citeonline{Grootendorst_2022}, surge como uma abordagem moderna. Seu diferencial reside na utilização dos \textit{embeddings} contextuais de modelos como o \gls{bert} para a modelagem de tópicos. Esta técnica permite identificar tópicos de forma dinâmica e mais coesa, superando as deficiências de modelos tradicionais ao capturar nuances semânticas e lidar com a complexidade de textos interdisciplinares.

Este projeto de pesquisa foca no desenvolvimento de um \textit{pipeline} computacional para o mapeamento interativo de publicações científicas. A proposta central é construir um artefato que combina a modelagem de tópicos do \gls{bertopic} \citeonline{Grootendorst_2022} com a ferramenta de visualização \gls{wizmap}\footnote{Disponível em: \url{https://github.com/poloclub/wizmap}.} \cite{Wang_2023}. O estudo de caso é aplicado ao acervo do Observatório de dados públicos de ciência e tecnologia da Bahia, que coleta informações de fontes como Currículos Lattes, Plataforma Sucupira e \textit{OpenAlex}, e tem um papel fundamental na gestão do conhecimento científico regional.

A metodologia \gls{dsr} é adotada como a estrutura principal deste estudo, orientando a criação deste artefato. O objetivo é transformar a base de dados textual do Observatório em um mapa de conhecimento navegável. Nessa solução, o \gls{bertopic} é empregado para extrair os padrões temáticos e o \gls{wizmap} é utilizado para a exploração visual e interativa desses tópicos. Essa integração permite a identificação de temas emergentes e a compreensão da estrutura do conhecimento científico da plataforma, indo além das análises estáticas tradicionais. Espera-se que este mecanismo otimize a experiência dos usuários e contribua para a gestão estratégica da pesquisa na plataforma.

Para articular o desenvolvimento deste estudo, a monografia segue uma progressão lógica. O Capítulo 2 estabelece o Referencial Teórico, fundamentando os conceitos de Ciência da Informação, a arquitetura dos Transformadores e as técnicas de modelagem de tópicos, com foco nos componentes do \gls{bertopic}. A seguir, o Capítulo 3 analisa os Trabalhos Correlatos, contextualizando esta pesquisa frente ao estado da arte. O Capítulo 4 detalha a Metodologia (\gls{dsr}), que fornece o rigor científico para a construção do artefato proposto. O Capítulo 5, cerne deste trabalho, apresenta o Projeto de Desenvolvimento, descrevendo a arquitetura completa do \textit{pipeline}: desde a ingestão dos dados do Observatório e a modelagem com \gls{bertopic}, até a integração final com a ferramenta de visualização interativa \gls{wizmap}. Por fim, o Capítulo 6 discute os Resultados Esperados e os métodos de validação aplicados a essa solução de mapeamento de conhecimento.