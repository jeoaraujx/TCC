\chapter[REFERENCIAL TEÓRICO]{REFERENCIAL TEÓRICO}
O referencial teórico deste estudo abordará diversos aspectos cruciais relacionados à Ciência da Informação, Análise de Publicações Científicas, Processamento de Linguagem Natural (PLN), Modelagem de Tópicos, e Modelos de Linguagem de Grande Escala (LLMs). 

\section{Ciência da Informação e Análise de Publicações Científicas}

A explosão da produção científica global nas últimas décadas, impulsionada pela maior acessibilidade à tecnologia e pela colaboração interdisciplinar, delineia um cenário desafiador para a área da Ciência da Informação. Como destacam \citeonline{Kim_2024}, o volume crescente de publicações dificulta a atualização contínua de pesquisadores e a identificação de áreas emergentes do conhecimento. Os autores reforçam essa problemática no resumo de seu trabalho:

\begin{citacao}
A produção científica global está se expandindo exponencialmente, o que, por sua vez, exige uma melhor compreensão da ciência da ciência e, especialmente, de como as fronteiras dos campos científicos se expandem através de processos de emergência. \citeonline[Traduzido, p. 1]{Kim_2024}
\end{citacao}

Nesse contexto, estratégias tradicionais de busca baseadas em palavras-chave mostram-se limitadas, uma vez que desconsideram a complexidade semântica do léxico científico. Esse fator resulta não apenas na omissão de trabalhos relevantes, mas também na dificuldade de mapear de forma consistente o progresso em determinados campos.

Um aspecto que amplia essa complexidade é a diversidade linguística no ambiente científico. Segundo \citeonline{Xie_2020}, embora o inglês desempenhe papel predominante, uma parcela significativa da produção ocorre em outros idiomas. Metodologias convencionais de análise revelam-se insuficientes para o tratamento multilíngue, o que pode restringir a circulação global do conhecimento e reduzir a visibilidade de estudos relevantes.

\begin{citacao}
A maioria dos estudos até hoje sobre análise de tópicos tem sido baseada em publicações em língua inglesa e tem dependido fortemente da análise de evolução de tópicos baseada em citações. [...] metodologias baseadas em citações não são adequadas para analisar relações de tópicos de pesquisa multilíngues. \citeonline[Traduzido, p. 1]{Xie_2020}
\end{citacao}

Diante desse cenário, técnicas contemporâneas de \textit{Topic Modeling}, em especial aquelas fundamentadas em \textit{embeddings}, têm sido investigadas como alternativas promissoras. De acordo com \citeonline{Galli_2024}, a utilização de representações densas derivadas de modelos como o \gls{bert} potencializa a análise de grandes volumes textuais, permitindo capturar aspectos semânticos que vão além da simples coincidência lexical.

\begin{citacao}
Um componente essencial para alcançar a compreensão semântica são os \textit{embeddings} — representações numéricas que codificam o significado de palavras ou mesmo de frases — que são fundamentais no \gls{pln} para capturar relacionamentos complexos entre palavras e frases usando arquiteturas especiais conhecidas como transformadores. \citeonline[Traduzido, p. 2]{Galli_2024}
\end{citacao}

Essa capacidade favorece a identificação de padrões temáticos em documentos que não compartilham necessariamente o mesmo vocabulário. Métodos modernos, como o \gls{bertopic}, oferecem uma estrutura metodológica para a extração de tópicos a partir dessas representações vetoriais densas. A literatura aponta que a aplicação dessas ferramentas é particularmente relevante em textos científicos heterogêneos e multilíngues, como os encontrados em grandes repositórios de publicações científicas, dada a sua robustez em capturar nuances semânticas independentemente do idioma \cite{Xie_2020}.

\section{Processamento de Linguagem Natural (PLN)}

O Processamento de Linguagem Natural (\gls{pln}) é um campo multidisciplinar, situado na interseção da Inteligência Artificial, da Linguística Computacional e da Ciência da Informação. O objetivo central da área é desenvolver métodos computacionais capazes de processar, analisar, compreender e gerar a linguagem humana — seja em formato de texto ou voz — de maneira funcional \cite{Jurafsky_2009}\footnote{Refere-se à obra \textit{Speech and Language Processing}, de Daniel Jurafsky e James H. Martin. É amplamente considerado o livro-texto acadêmico padrão e a referência canônica para o ensino e estudo do Processamento de Linguagem Natural em todo o mundo.}.

Historicamente, o \gls{pln} fundamentava-se em abordagens estatísticas e regras linguísticas manuais para modelar a linguagem, conforme descrito por \cite{Manning_1999}\footnote{Refere-se à obra \textit{Foundations of Statistical Natural Language Processing} (Manning e Schütze, 1999), considerada o trabalho seminal que consolidou as abordagens estatísticas como o padrão do \gls{pln} antes da ascensão das redes neurais profundas.}. As técnicas de \gls{pln} são projetadas para extrair significado e estrutura de dados textuais, que são inerentemente não estruturados. Este processo envolve uma série de tarefas que variam desde a análise sintática (a estrutura gramatical) até a análise semântica (o significado por trás das palavras). Entre as aplicações comuns estão a classificação de textos, a tradução automática, a sumarização de documentos e a Modelagem de Tópicos (\textit{Topic Modeling}), que é o foco desta pesquisa.

A evolução recente do campo foi impulsionada pelo \textit{Deep Learning} (Aprendizado Profundo), que permitiu a criação de representações vetoriais mais precisas. Como destacam \citeonline{Galli_2024}, o \gls{pln} moderno passou a depender da capacidade de capturar o significado contextual, superando a análise baseada apenas na contagem de palavras. Essa transição para uma abordagem focada na compreensão semântica viabilizou os avanços recentes em modelagem de tópicos.

\section{A Evolução das Representações Vetoriais em PLN}

O progresso na área de \gls{pln} tem sido caracterizado pela investigação de representações vetoriais capazes de capturar não apenas a estrutura sintática, mas também os aspectos semânticos e contextuais dos textos. A evolução dessas representações partiu de abordagens estáticas para modelos dinâmicos baseados em contexto.

\subsection{Embeddings Estáticos: Limitações do Bag-of-Words}

As primeiras abordagens de sucesso, como o \textit{Word2Vec} proposto por \citeonline{Mikolov_2013}\footnote{O \textit{Word2Vec} (2013) foi seminal por introduzir duas arquiteturas eficientes, \textit{Skip-gram} e \textit{CBOW}, que aprendem vetores de palavras prevendo o contexto em que elas aparecem, baseando-se na hipótese distribucional.} e o \textit{GloVe} proposto por \citeonline{Pennington_2014}\footnote{O \textit{GloVe} (2014), ou \enquote{Global Vectors}, diferencia-se por combinar as estatísticas globais de coocorrência de palavras (como o LSA) com a modelagem baseada em janelas de contexto (como o \textit{Word2Vec}), capturando relações lineares entre palavras.}, consolidaram o conceito de \textit{embeddings}. Nesse caso, a operação algébrica subtrai o vetor de \enquote{Homem} do vetor de \enquote{Rei}, isolando o conceito de realeza, e adiciona o vetor de \enquote{Mulher}, resultando em uma representação vetorial espacialmente próxima à de \enquote{Rainha}. Isso demonstra que o modelo é capaz de codificar conceitos abstratos, como gênero, através da direção e distância entre os vetores. Estes consistem em vetores em espaços de alta dimensionalidade capazes de representar o significado aproximado de uma palavra.

A contribuição desses modelos foi permitir a quantificação do significado semântico. Em vez de tratar palavras como identificadores discretos (como em uma abordagem \textit{bag-of-words}), os \textit{embeddings} posicionam termos com significados similares próximos uns dos outros no espaço vetorial. Isso permite que relações semânticas sejam capturadas matematicamente, como no exemplo clássico \enquote{Rei - Homem + Mulher $\approx$ Rainha} \cite{Mikolov_2013}. \citeonline{Xie_2020} na literatura de \gls{pln} refere-se a este espaço vetorial como um \enquote{espaço semântico}.

O aprendizado desses vetores ocorre através do treinamento de redes neurais em tarefas de previsão de contexto, conforme ilustrado na Figura \ref{fig:cbow_skipgram}. O artigo seminal de \citeonline{Mikolov_2013} introduziram duas arquiteturas principais:

\begin{enumerate}
    \item \textbf{\gls{cbow}:} A arquitetura prevê a palavra atual (saída) com base em uma janela de palavras do contexto (entrada).
    \item \textbf{\textit{Skip-gram}:} A arquitetura inverte a lógica e usa a palavra atual (entrada) para prever as palavras do contexto (saída).
\end{enumerate}

É importante destacar que os \textit{embeddings} não são o produto final, mas sim um subproduto do treinamento: os vetores aprendidos na camada oculta da rede (\textit{PROJECTION} na figura) tornam-se a representação semântica da palavra, como indica \citeonline[p. 4]{Mikolov_2013}.

\begin{figure}[H]
    \centering
    \includegraphics[width=0.7\textwidth]{figs/Cbow_skipgram.png} 
    \caption{Arquiteturas \gls{cbow} e \textit{Skip-gram}.}
    \label{fig:cbow_skipgram}
    \legend{Fonte: \citeonline[p. 5]{Mikolov_2013}}
\end{figure}

Apesar da utilidade em capturar similaridades lexicais, esses modelos apresentavam a limitação de atribuir um único vetor fixo a cada termo, independentemente do contexto de ocorrência. Por exemplo, a palavra \enquote{banco} teria a mesma representação vetorial em \enquote{banco financeiro} e \enquote{banco da praça}. Tal restrição, usualmente referida como o problema da ambiguidade do significado da palavra (\textit{ambiguity of word meaning}), compromete a precisão em tarefas que exigem desambiguação semântica.

\subsection{A Revolução dos Transformadores e o Mecanismo de Atenção}

Uma mudança significativa no paradigma ocorreu com a introdução do modelo de Transformadores (\textit{Transformers}), proposto por \citeonline{vaswani_2017} no artigo seminal \textit{Attention Is All You Need}\footnote{Este artigo é considerado um dos trabalhos mais influentes da \gls{pln} moderna. Sua principal contribuição foi propor uma arquitetura de rede neural que dispensa totalmente as camadas recorrentes (RNN) e convolucionais, baseando-se unicamente em mecanismos de atenção para modelar dependências globais entre a entrada e a saída \cite[p. 1]{vaswani_2017}.}. Essa arquitetura diferencia-se das \gls{rnn} e convolucionais, por fundamentar-se inteiramente no mecanismo de autoatenção (\textit{self-attention}).

Por meio da autoatenção, o modelo atribui pesos diferenciados a \textit{tokens} em uma sequência, permitindo processar simultaneamente e de forma bidirecional a totalidade do contexto textual. A arquitetura do Transformador, conforme apresentado na Figura \ref{fig:transformer}, segue uma estrutura de codificador-decodificador (\textit{encoder-decoder}). O lado esquerdo do diagrama representa o Codificador, enquanto o lado direito representa o Decodificador.

\begin{figure}[H]
    \centering
    \includegraphics[width=0.4\textwidth]{figs/Transformadores.png} 
    \caption{Arquitetura do modelo Transformador.}
    \label{fig:transformer}
    \legend{Fonte: \citeonline[p. 3]{vaswani_2017}} 
\end{figure}


O \textbf{Codificador} (\textit{Encoder}) é composto por uma pilha de N camadas idênticas (no artigo original, N=6). Cada camada, por sua vez, é composta por duas subcamadas principais: um mecanismo de autoatenção \textit{multi-head} (\textit{multi-head self-attention}) e uma rede neural \textit{feed-forward} (rede neural de alimentação direta) simples e totalmente conectada. Conexões residuais seguidas de normalização de camada (\textit{Add \& Norm}) são aplicadas ao redor de cada subcamada.

O \textbf{Decodificador} (\textit{Decoder}), de forma similar, é uma pilha de N camadas. Além das duas subcamadas presentes no codificador, o decodificador insere uma terceira subcamada, que realiza a atenção \textit{multi-head} sobre a saída da pilha do codificador. Crucialmente, a subcamada de autoatenção do decodificador é \enquote{mascarada} (\textit{Masked Multi-Head Attention}). Esse mascaramento é o que garante que a previsão para uma posição \textit{i} só possa depender das saídas conhecidas em posições anteriores a \textit{i}, preservando a propriedade autorregressiva do modelo.

Embora a arquitetura completa do Transformador tenha sido projetada para tarefas de transdução de sequência (como a tradução automática), foi a sua pilha de \textbf{Codificadores} (\textit{Encoder}) que se mostrou revolucionária para tarefas de \textit{compreensão} de linguagem. A capacidade do Codificador de processar texto de forma bidirecional e gerar representações numéricas ricas em contexto estabeleceu a base para uma nova classe de modelos focados exclusivamente na representação semântica, como será detalhado a seguir.

\subsection{Embeddings Contextuais: BERT e SBERT}
\label{sec:bert_sbert}

Sobre a base arquitetônica dos Transformadores, foram desenvolvidos os modelos pré-treinados, entre os quais se destaca o \gls{bert}, introduzido por \citeonline{Devlin_2019}. O \gls{bert} utiliza a arquitetura do Codificador (\textit{Encoder}) do Transformador para gerar representações de linguagem.

A inovação fundamental do \gls{bert} foi o pré-treinamento bidirecional, que diferentemente de abordagens anteriores, como o \gls{gpt} de \citeonline{Radford_2018}, que utilizava um treinamento unidirecional (da esquerda para a direita), o \gls{bert} foi projetado para \enquote{pré-treinar representações profundamente bidirecionais, condicionando conjuntamente o contexto esquerdo e direito em todas as camadas} como apontam \citeonline[p. 1, Traduzido]{Devlin_2019}.

Para alcançar essa bidirecionalidade sem que o modelo \enquote{visse a resposta}, \citeonline{Devlin_2019} introduziu o objetivo do \gls{mlm}\footnote{O \gls{mlm} é inspirado na tarefa \textit{Cloze} \cite{Taylor_1953}, onde o modelo deve prever palavras que foram omitidas (mascaradas) de uma sentença, usando o contexto de ambas as direções (esquerda e direita) para fazer a previsão \cite[p. 1]{Devlin_2019}.}. A Figura \ref{fig:bert_vs_gpt} ilustra a diferença fundamental entre as arquiteturas de pré-treinamento, mostrando como o \gls{bert} é capaz de processar informações de toda a sequência em todas as suas camadas.

\begin{figure}[H]
    \centering
    \includegraphics[width=0.7\textwidth]{figs/Bert_vs_gpt.png} 
    \caption{Diferenças nas arquiteturas de pré-treinamento. \gls{bert} é bidirecional, \gls{gpt} é unidirecional (da esquerda para a direita) e \gls{elmo}.}
    \label{fig:bert_vs_gpt}
    \legend{Fonte: \citeonline[p. 13]{Devlin_2019}}
\end{figure}

Apesar do desempenho em tarefas de classificação, a arquitetura do original do \gls{bert} apresentou limitações para tarefas de busca de similaridade semântica ou \textit{clustering}. Conforme observador por \citeonline{Reimers_2019}, o uso do \gls{bert} \enquote{requer que ambas as sentenças sejam alimentadas na rede, o que causa um overhead computacional massivo}. Uma busca de similaridade em 10.000 sentenças, por exemplo, exigiria cerca de 50 milhões de inferências (aproximadamente 65 horas), tornando-o inviável para grandes bases de dados. Além disso, estudos empíricos demonstraram que usar os \textit{embeddings} \enquote{crus} do \gls{bert} (seja pela média das saídas ou pelo vetor do \textit{token} `[CLS]`) produz resultados insatisfatórios, muitas vezes piores do que os \textit{embeddings} estáticos como o \textit{GloVe}.

Para solucionar essa questão, \citeonline{Reimers_2019} propuseram o \gls{sbert}. Ele modifica o \gls{bert} pré-treinado, adicionando uma operação de \textit{pooling} (sendo a média, \textit{MEAN-strategy}, a mais comum) à saída do \gls{bert} para criar um \textit{embedding} de sentença de tamanho fixo.

Crucialmente, o \gls{sbert} utiliza redes siamesas\footnote{Redes siamesas são uma arquitetura onde duas ou mais redes neurais idênticas (com pesos compartilhados) processam entradas diferentes de forma independente. Elas são otimizadas para aprender uma função de similaridade, aproximando os vetores de saída para entradas similares e afastando-os para entradas diferentes.} para fazer o \textit{fine-tuning} desses \textit{embeddings} de sentença. A Figura \ref{fig:sbert_arch} ilustra a arquitetura de inferência do \gls{sbert}, onde duas sentenças (A e B) são processadas por redes \gls{bert} idênticas (com pesos compartilhados), gerando vetores de sentença \textbf{u} e \textbf{v}. Esses vetores podem, então, ser comparados eficientemente usando uma medida de similaridade, como a similaridade de cosseno (\textit{cosine-similarity}).

\begin{figure}[H]
    \centering
    \includegraphics[width=0.7\textwidth]{figs/Sbert_arch.png} 
    \caption{Arquitetura de inferência do \gls{sbert} para computar similaridade.}
    \label{fig:sbert_arch}
    \legend{Fonte: Adaptado de \citeonline[p. 3]{Reimers_2019}}
\end{figure}

\citeonline{Reimers_2019} demonstraram que essa abordagem reduz o custo computacional de encontrar o par mais similar em 10.000 sentenças de 65 horas (com \gls{bert}) para cerca de 5 segundos. Essa otimização para similaridade de sentenças permite o uso eficiente desses vetores em cenários multilíngues. A utilização de modelos pré-treinados em múltiplos idiomas (como o \textit{paraphrase-multilingual-MiniLM-L12-v2}\footnote{Disponível em: \url{https://huggingface.co/sentence-transformers/paraphrase-multilingual-MiniLM-L12-v2}.}) torna-se particularmente relevante, visto que tais modelos produzem \textit{embeddings} semanticamente consistentes mesmo em diferentes idiomas.

\subsection{Abordagens Tradicionais de Modelagem de Tópicos}

Com o crescimento exponencial de dados textuais e a consequente necessidade de organizar informação em larga escala, a modelagem de tópicos consolidou-se como uma técnica fundamental na área de \gls{pln}. Em termos gerais, trata-se de um conjunto de métodos estatísticos cujo objetivo é identificar estruturas semânticas latentes\footnote{O termo "latente" significa que os tópicos não são diretamente observáveis, mas sim inferidos estatisticamente a partir dos padrões de coocorrência de palavras no \textit{corpus}.} — denominadas \textit{tópicos} — em coleções de documentos. Assim, essas técnicas permitem inferir distribuições temáticas que não são explicitamente observáveis, mas que emergem a partir de regularidades no uso do vocabulário.

Entre as abordagens iniciais destacam-se três marcos históricos: a \gls{lsa}, a \gls{plsa} e a \gls{lda}. Esses métodos não apenas moldaram a compreensão inicial sobre a representação semântica de textos, como também estabeleceram fundamentos conceituais e metodológicos que orientaram o desenvolvimento de modelos mais avançados.

A \gls{lsa}, proposta por \citeonline{Deerwester_1990}, parte da decomposição de matrizes termo-documento por meio da técnica de \gls{svd}\footnote{A \gls{svd} é uma técnica de álgebra linear para a decomposição de matrizes que permite encontrar a melhor aproximação de uma matriz por outra de posto inferior, sendo fundamental para a redução de dimensionalidade em espaços vetoriais de termos.}. Nesse enquadramento, documentos e termos são projetados em um espaço vetorial de dimensionalidade reduzida, o que permite atenuar ruídos lexicais e capturar relações de similaridade latentes. Apesar de sua relevância histórica, a linearidade da \gls{lsa} e sua insensibilidade a variações contextuais limitam seu desempenho em cenários onde relações semânticas complexas são determinantes \cite{George_2023, Xie_2020}.

Com o intuito de superar parte dessas limitações, \citeonline{Hofmann_1999,Hofmann_2001} introduziram a \gls{plsa}, que reformulou a representação semântica a partir de um modelo probabilístico. Nessa abordagem, cada ocorrência de palavra em um documento é modelada como proveniente de um tópico latente, de forma que a probabilidade conjunta de palavra $w$ e documento $d$ é expressa como:
\[
P(w, d) = \sum_{z \in Z} P(z|d) \, P(w|z),
\]
onde $z$ representa o conjunto de tópicos latentes. Embora tenha representado um avanço em relação à \gls{lsa}, a \gls{plsa} apresenta limitações notáveis, em especial no que se refere à escalabilidade: o número de parâmetros cresce linearmente com a quantidade de documentos, o que compromete sua generalização e a torna suscetível a \textit{overfitting} \cite{Datchanamoorthy_2023}.

A evolução natural desse paradigma ocorreu com a formulação da \gls{lda}, proposta por \citeonline{Blei_2003}. Ao contrário da \gls{plsa}, a \gls{lda} incorpora uma camada Bayesiana por meio da utilização de distribuições de \textit{Dirichlet} como \textit{priors}\footnote{A \gls{lda} é um modelo generativo Bayesiano. O uso das distribuições de \textit{Dirichlet} (uma distribuição de probabilidade sobre outras distribuições) permite ao modelo tratar as misturas de tópicos nos documentos e as misturas de palavras nos tópicos como variáveis aleatórias, conferindo maior robustez e melhor generalização.}. Essa estrutura permite regularizar o modelo e definir uma distribuição de tópicos não apenas a nível de documento, mas também a nível de \textit{corpus}, resultando em maior robustez e interpretabilidade. A \gls{lda} parte da premissa de que cada documento é representado como uma mistura de tópicos, e cada tópico, por sua vez, é caracterizado por uma distribuição de palavras. Essa formulação tornou o modelo amplamente aplicável em diferentes domínios, como saúde pública \cite{Mifrah_2020} e eficiência energética \cite{Polyzos_2022}.

Apesar de sua influência, tanto a \gls{lsa} quanto a \gls{plsa} e a \gls{lda} compartilham limitações estruturais. Todas operam no paradigma de \textit{bag-of-words}\footnote{O \textit{Bag-of-Words} (Saco de Palavras) é um modelo de representação de texto que ignora a ordem e a estrutura gramatical das palavras, tratando um documento apenas como um conjunto (ou multiconjunto) de suas palavras e suas frequências.} (saco de palavras), que ignora a ordem e o contexto local das palavras. Segundo \citeonline{George_2023, Xie_2020}, isso frequentemente conduz a representações semânticas superficiais em textos técnicos ou multilíngues. \citeonline{Datchanamoorthy_2023} também reitera que a sensibilidade da \gls{lda} à definição do número de tópicos ($K$) representa um desafio adicional: valores reduzidos podem fundir tópicos distintos em um único, enquanto valores elevados podem fragmentar temas coesos em subtemas artificiais.

\begin{citacao}
A sensibilidade do \gls{lda} ao parâmetro do número de temas ($K$) é uma de suas desvantagens. Encontrar o valor ideal para ($K$) pode ser desafiador. O modelo pode simplificar excessivamente e combinar diferentes temas em um só se ($K$) for configurado muito baixo. No entanto, se ($K$) for configurado muito alto, o modelo pode se tornar muito complexo e produzir temas errôneos \cite[Tradução nossa]{Datchanamoorthy_2023}.
\end{citacao}

Essas restrições evidenciam que, embora fundamentais, tais técnicas falham em capturar o significado contextual profundo e a ordem das palavras. Essa limitação estrutural torna-os insuficientes para tarefas que exigem uma compreensão semântica robusta, especialmente em bases textuais heterogêneos ou multilíngues onde a ambiguidade lexical é alta, destacando a necessidade de abordagens que superem o paradigma \textit{bag-of-words}.

% \section{Abordagens Tradicionais de Modelagem de Tópicos}

% Com o crescimento exponencial de dados textuais e a consequente necessidade de organizar informação em larga escala, a modelagem de tópicos consolidou-se como uma técnica fundamental na área de \gls{PLN}. Em termos gerais, trata-se de um conjunto de métodos estatísticos cujo objetivo é identificar estruturas semânticas latentes — denominadas \textit{tópicos} — em coleções de documentos. Assim, essas técnicas permitem inferir distribuições temáticas que não são explicitamente observáveis, mas que emergem a partir de regularidades no uso do vocabulário. Essa perspectiva abriu caminho para aplicações em áreas diversas, desde ciências sociais até biomedicina (\citeonline{Jung_2024}).

% Entre as abordagens iniciais destacam-se três marcos históricos: a \textit{Latent Semantic Analysis} \gls{LSA}, a \textit{Probabilistic Latent Semantic Analysis} \gls{PLSA} e a \textit{Latent Dirichlet Allocation} \gls{LDA}. Esses métodos não apenas moldaram a compreensão inicial sobre a representação semântica de textos, como também estabeleceram fundamentos conceituais e metodológicos que orientaram o desenvolvimento de modelos mais avançados.

% A \gls{LSA}, proposta por \citeonline{Deerwester_1990}, parte da decomposição de matrizes termo-documento por meio da técnica de \textit{Singular Value Decomposition} (\gls{SVD}). Nesse enquadramento, documentos e termos são projetados em um espaço vetorial de dimensionalidade reduzida, o que permite atenuar ruídos lexicais e capturar relações de similaridade latentes. Apesar de sua relevância histórica, a linearidade da \gls{LSA} e sua insensibilidade a variações contextuais limitam seu desempenho em cenários onde relações semânticas complexas são determinantes (\citeonline{George_2023, Xie_2020}).

% Com o intuito de superar parte dessas limitações, \citeonline{Hofmann_1999,Hofmann_2001} introduziram a \gls{PLSA}, que reformulou a representação semântica a partir de um modelo probabilístico. Nessa abordagem, cada ocorrência de palavra em um documento é modelada como proveniente de um tópico latente, de forma que a probabilidade conjunta de palavra $w$ e documento $d$ é expressa como:
% \[
% P(w, d) = \sum_{z \in Z} P(z|d) \, P(w|z),
% \]
% onde $z$ representa o conjunto de tópicos latentes. Embora tenha representado um avanço em relação à \gls{LSA}, a \gls{PLSA} apresenta limitações notáveis, em especial no que se refere à escalabilidade: o número de parâmetros cresce linearmente com a quantidade de documentos, o que compromete sua generalização e a torna suscetível a \textit{overfitting} (\citeonline{Datchanamoorthy_2023}).

% A evolução natural desse paradigma ocorreu com a formulação da \gls{LDA}, proposta por \citeonline{Blei_2003}. Ao contrário da \gls{PLSA}, a \gls{LDA} incorpora uma camada Bayesiana por meio da utilização de distribuições de \textit{Dirichlet} como \textit{priors}. Essa estrutura permite regularizar o modelo e definir uma distribuição de tópicos não apenas a nível de documento, mas também a nível de corpus, resultando em maior robustez e interpretabilidade. A \gls{LDA} parte da premissa de que cada documento é representado como uma mistura de tópicos, e cada tópico, por sua vez, é caracterizado por uma distribuição de palavras. Essa formulação tornou o modelo amplamente aplicável em diferentes domínios, como saúde pública (\citeonline{Mifrah_2020}) e eficiência energética (\citeonline{Polyzos_2022}).

% Apesar de sua influência, tanto a \gls{LSA} quanto a \gls{PLSA} e a \gls{LDA} compartilham limitações estruturais. Todas operam no paradigma de \textit{bag-of-words}, que ignora a ordem e o contexto local das palavras, o que frequentemente conduz a representações semânticas superficiais em textos técnicos ou multilíngues (\citeonline{George_2023, Xie_2020}). Além disso, a sensibilidade da \gls{LDA} à definição do número de tópicos ($K$) representa um desafio adicional: valores reduzidos podem fundir tópicos distintos em um único, enquanto valores elevados podem fragmentar temas coesos em subtemas artificiais (\citeonline{Datchanamoorthy_2023}).

% \begin{citacao}
% A sensibilidade do LDA ao parâmetro do número de temas ($K$) é uma de suas desvantagens. Encontrar o valor ideal para ($K$) pode ser desafiador. O modelo pode simplificar excessivamente e combinar diferentes temas em um só se ($K$) for configurado muito baixo. No entanto, se ($K$) for configurado muito alto, o modelo pode se tornar muito complexo e produzir temas errôneos (\citeonline[Traduzido]{Datchanamoorthy_2023}).
% \end{citacao}

% Essas restrições evidenciam que, embora fundamentais, tais técnicas não capturam relações profundas e não lineares entre palavras e tópicos. Esse cenário motivou a emergência de abordagens modernas baseadas em \textit{embedding} e arquiteturas de \textit{transformer} (\citeonline{vaswani_2017, Devlin_2019, Radford_2018}), que oferecem maior sensibilidade contextual e escalabilidade para corpora heterogêneos e de grande volume.

% \section{BERTopic: Uma Abordagem Moderna}

% O \textit{Bidirectional Encoder Representations from Transformers} \gls{BERT}, introduzido por \citeonline{Devlin_2019}, marcou um avanço significativo no campo do \textit{Natural Language Processing} \gls{NLP}. Baseado na arquitetura de \textit{Transformers} (\citeonline{vaswani_2017}), o BERT emprega o mecanismo de \textit{self-attention} para capturar relações contextuais entre palavras em um texto. Diferentemente de abordagens anteriores, que analisavam sequências de maneira unidirecional, o BERT considera simultaneamente o contexto à esquerda e à direita de cada palavra, resultando em \textit{embeddings} ricos e contextuais. Essa característica tornou o BERT amplamente utilizado em tarefas como classificação de texto, análise de sentimentos e resposta a perguntas.

% Apesar de sua relevância, o BERT não foi projetado para tarefas de similaridade semântica entre sentenças ou documentos, pois os vetores que gera não são diretamente comparáveis em termos de proximidade semântica (\citeonline{Reimers_2019}). Essa limitação levou ao desenvolvimento do \textit{Sentence-BERT} \gls{S-BERT}, uma variante que adapta o BERT ao treinamento em redes siamesas (\textit{Siamese Networks}) e funções de perda específicas, como \textit{triplet loss}. O resultado é a produção de \textit{embeddings} que são calculados por meio de técnicas como \textit{Class-based Term Frequency-Inverse Document Frequency} (c-TF-IDF), que ajusta os pesos das palavras com base em suas frequências e relevâncias dentro de um corpus, elas podem ser comparadas de forma eficiente por meio de medidas de distância, como \textit{cosine similarity}, viabilizando tarefas de busca semântica e agrupamento de documentos.

% Sobre essa base, \citeonline{Grootendorst_2022} propôs o \textit{BERTopic}, que não deve ser entendido como um único modelo, mas como um \textit{pipeline} que integra diferentes técnicas complementares para a modelagem de tópicos. Esse arranjo inicia-se pela geração de \textit{embeddings} com o S-BERT, etapa que garante representações semânticas adequadas para comparação entre artigos científicos. Essa combinação permite que o BERTopic identifique tópicos de maneira dinâmica e precisa, mesmo em grandes volumes de dados textuais diversificados (\citeonline{George_2023, Jung_2024, Datchanamoorthy_2023}).

% \begin{figure}[h]
%     \centering
%     \caption{\label{bertopic-eschema}Diagrama esquemático detalhado da comparação de métricas de avaliação entre modelos.}
%     \includegraphics[scale=0.8]{figs/Bertopic.png}
%     \caption*{\footnotesize Fonte: (\citeonline[p. 7, Tradução nossa]{Jung_2024})}
% \end{figure}

% Como observado no diagrama comparativo entre modelos, a etapa de redução de dimensionalidade no pipeline utiliza o \textit{Uniform Manifold Approximation and Projection} \gls{UMAP} (\citeonline{McInnes_2018}, uma técnica que projeta vetores de alta dimensionalidade em um espaço reduzido. Essa abordagem se fundamenta em princípios teóricos de geometria Riemanniana e topologia algébrica , o que a diferencia de métodos mais antigos, como o \textit{t-SNE} (\citeonline{Maaten_2008}), e lhe confere maior escalabilidade e eficiência para a análise de grandes volumes de dados. O UMAP opera em duas fases principais: primeiro, constrói um grafo ponderado que representa a estrutura topológica dos dados em alta dimensão; em seguida, projeta esse grafo para um espaço de baixa dimensão, otimizando o layout para minimizar a entropia cruzada entre as duas representações. Essa metodologia é crucial para preservar tanto as estruturas locais quanto as globais do corpus, garantindo a coesão semântica dos dados. Ao aplicar o UMAP ao conjunto de publicações científicas da plataforma SIMCC, é possível manter a fidelidade das relações entre os documentos, um requisito fundamental para a subsequente fase de agrupamento do BERTopic.

% \begin{figure}[h]
%     \centering
%     \includegraphics[width=0.5\linewidth]{figs/UMAP.png}
%     \caption{Diagrama ilustrativo do UMAP, que demonstra a relação entre os hiperparâmetros \textit{n\_neighbors} e \textit{min\_dist} e a representação visual dos dados. O parâmetro \textit{n\_neighbors} controla a balança entre a preservação da estrutura global (valores altos) e local (valores baixos), enquanto \textit{min\_dist} ajusta a densidade dos agrupamentos, determinando a proximidade entre os pontos no espaço de baixa dimensionalidade. Esta visualização é crucial para otimizar o algoritmo e garantir que a estrutura semântica das publicações científicas seja fielmente representada para a subsequente etapa de agrupamento.}
%     \caption*{\footnotesize Fonte: (\citeonline[p. 24]{McInnes_2018})}
%     \label{fig:UMAP}
% \end{figure}

% Com os vetores de alta dimensionalidade reduzidos pelo UMAP, a etapa subsequente é o agrupamento por meio do \textit{Hierarchical Density-Based Spatial Clustering of Applications with Noise} \gls{HDBSCAN}. Diferentemente de métodos clássicos como o K-Means, que assume \textit{clusters} esféricos e de densidade uniforme, o HDBSCAN é um algoritmo de agrupamento baseado em densidade que não faz suposições prévias sobre a forma ou a densidade dos agrupamentos (\citeonline{Campello_2013}). Sua arquitetura hierárquica constrói uma árvore de conectividade que reflete a estrutura de densidade subjacente dos dados, permitindo a identificação de \textit{clusters} de densidade variável. Essa capacidade é particularmente relevante para a análise de publicações científicas, onde a distribuição dos tópicos tende a ser heterogênea. O HDBSCAN também se destaca por sua robustez ao tratar documentos que não se ajustam a nenhum padrão temático, classificando-os como outliers de forma intrínseca, sem a necessidade de um passo de pós-processamento. Essa característica é especialmente relevante em contextos de produção científica, onde coexistem tanto publicações centrais com alta densidade de tópicos quanto trabalhos periféricos ou com temas emergentes. Essa abordagem garante que a sua análise não apenas identifique os tópicos dominantes, mas também lide eficientemente com a diversidade e o ruído natural do corpus da plataforma SIMCC.

% \begin{figure}
%     \centering
%     \includegraphics[width=0.5\linewidth]{figs/HDBSCAN.png}
%     \caption{Figura ilustrativa de um \textit{dataset} sintético com quatro \textit{clusters} e ruído de fundo. A imagem demonstra o tipo de desafio que o algoritmo HDBSCAN é capaz de superar, como a identificação de agrupamentos de densidades e formas variadas, além de tratar \textit{outliers} de forma eficiente. Este comportamento é ideal para a análise de publicações científicas, onde a distribuição dos tópicos tende a ser heterogênea e não segue padrões geométricos rígidos.}
%     \caption*{\footnotesize Fonte: (\citeonline[p. 16]{Campello_2013})}
%     \label{fig:HDBSCAN}
% \end{figure}

% Por fim, o BERTopic aplica o \textit{class-based Term Frequency-Inverse Document Frequency} (c-TF-IDF), que trata cada cluster como um único documento. Essa abordagem destaca termos distintivos de cada grupo, permitindo identificar palavras-chave representativas mesmo quando não são as mais frequentes (\citeonline{Gana_2024, Grootendorst_2022}). O c-TF-IDF, portanto, fornece uma base interpretável para a descrição de cada tópico.

% A combinação dessas etapas — \textit{embeddings} com S-BERT, redução de dimensionalidade com UMAP, clusterização com HDBSCAN e representação com c-TF-IDF — estabelece um fluxo robusto para a modelagem de tópicos. No contexto deste trabalho, esse \textit{pipeline} constitui o núcleo do processo de análise das publicações científicas indexadas na plataforma SIMCC, servindo de ponto de partida para a integração com modelos de linguagem de grande escala, como o GPT-4, que será empregado para enriquecer semanticamente os rótulos dos tópicos e aprimorar sua interpretabilidade.

% \section{Modelos de Linguagem de Grande Escala (LLMs)}

% Os Modelos de Linguagem de Grande Escala (LLMs) constituem um marco no avanço do Processamento de Linguagem Natural (PLN), permitindo análises textuais sofisticadas e interpretações semânticas em volumes de dados sem precedentes. Fundamentados em arquiteturas baseadas em \textit{transformers}, como o BERT, GPT e suas variantes, esses modelos utilizam aprendizado profundo para construir representações contextuais dinâmicas de palavras e sentenças. Ao transformar o texto em \textit{embeddings} semânticos, capturam relações latentes complexas entre elementos linguísticos, servindo de alicerce para tarefas como sumarização automática, classificação de documentos e modelagem de tópicos (\citeonline{Meng_2024, Gana_2024}). 

% Enquanto LLMs podem ser definidos de forma geral como sistemas de PLN capazes de aprender distribuições linguísticas a partir de grandes corpora não anotados, o modo como tais modelos realizam o pré-treinamento e o ajuste fino (\textit{fine-tuning}) difere significativamente entre arquiteturas. As primeiras tentativas de modelos sequenciais, como Redes Neurais Recorrentes (RNNs) e Redes de Memória de Longo-Curto Prazo (LSTMs), apresentavam limitações na captura de dependências de longo alcance. Esse problema foi mitigado com a introdução do \textit{Transformer} por \citeonline{vaswani_2017}, cuja operação se baseia no Mecanismo de Atenção (\textit{self-attention}, permitindo atribuir diferentes pesos às palavras do contexto e, consequentemente, capturar relações semânticas globais de maneira mais eficiente.

% O treinamento de LLMs ocorre tipicamente em duas etapas complementares. Na fase de \textit{pré-treinamento}, emprega-se aprendizado não supervisionado para expor o modelo a trilhões de palavras em dados textuais como documentos e dados da internet, consolidando padrões gerais da linguagem. Dois paradigmas se destacam nesse processo: (i) a Modelagem de Linguagem Autorregressiva, como no GPT, onde o modelo aprende a prever o próximo token a partir de uma sequência de tokens anteriores; e (ii) a Modelagem de Linguagem Mascarada, como no BERT, em que lacunas são ocultadas e o modelo deve inferi-las a partir do contexto bidirecional (\citeonline{Devlin_2019, Jung_2024}). Em seguida, ocorre o \textit{fine-tuning}, etapa supervisionada em que o modelo é ajustado a tarefas específicas, como classificação de textos, análise de sentimentos ou sumarização, garantindo robustez e especialização (\cite{Gana_2024}).

% \begin{itemize}
%     \item \textbf{Modelagem de Linguagem Autorregressiva:} Modelos como o GPT (\textit{Generative Pre-trained Transformer}) seguem um fluxo sequencial unidirecional, prevendo cada token com base nos anteriores. Essa abordagem favorece a coerência narrativa e a fluidez na geração textual, aspectos essenciais em tarefas de criação de conteúdo (\citeonline{Radford_2018, Jung_2024}).
    
%     \item \textbf{Modelagem de Linguagem Mascarada:} Modelos como o BERT (\textit{Bidirectional Encoder Representations from Transformers}) aplicam mascaramento aleatório em tokens, forçando o modelo a interpretar o contexto bidirecionalmente. Tal característica possibilita uma maior sensibilidade semântica, útil em tarefas como inferência textual e modelagem de tópicos (\citeonline{Devlin_2019, Datchanamoorthy_2023}).
% \end{itemize}

% Entre os LLMs mais avançados, destaca-se o \textbf{GPT-4}, evolução do \textbf{GPT-1} desenvolvido pela OpenAI por \citeonline{Radford_2018}, que introduziu sua arquitetura a partir de um trabalho seminal. Sua estrutura permanece fundamentada no paradigma \textit{transformer} \citeonline{vaswani_2017}, mas incorpora modificações substanciais em relação às versões anteriores. Embora a documentação oficial seja limitada por razões proprietárias, o \textit{Technical Report} da OpenAI \cite{OpenAI_2023} e análises independentes \citeonline{Liu_2023, Achiam_2023} sugerem que o GPT-4 conta com bilhões de parâmetros adicionais em comparação ao GPT-3, além de maior profundidade de camadas de atenção e mecanismos otimizados de paralelização no treinamento distribuído. Essas melhorias resultam em avanços na capacidade de raciocínio semântico, na robustez diante de contextos ambíguos e na generalização para tarefas pouco definidas.

% Outro aspecto relevante é o aprimoramento nos métodos de alinhamento e segurança (\textit{alignment}), alcançados por meio de técnicas como o \textit{Reinforcement Learning with Human Feedback} (RLHF), que possibilitam ao modelo produzir respostas mais consistentes com critérios humanos de qualidade e relevância (\citeonline{OpenAI_2023, Ouyang_2022}). Além disso, o GPT-4 demonstra melhor desempenho em cenários multilíngues e em tarefas de alto nível cognitivo, como resolução de problemas em exames padronizados e síntese de conhecimento interdisciplinar (\citeonline{Achiam_2023}). Essas características tornam o modelo especialmente adequado para aplicações acadêmicas e científicas, onde a precisão semântica e a interpretabilidade das respostas são fundamentais.

% Ao compararmos o \textbf{BERTopic} e o GPT-4, evidenciam-se diferenças fundamentais na natureza e aplicação de cada modelo. O BERTopic, embora baseado em \textit{embeddings} derivados de modelos como BERT, concentra-se em identificar e organizar tópicos latentes a partir de representações vetoriais de documentos, utilizando algoritmos já citados nas seções anteriores. Seu ponto forte está na capacidade de estruturar grandes volumes de dados em \textit{clusters} semanticamente coerentes \citeonline{Grootendorst_2022}. Já o GPT-4, além de gerar representações contextuais sofisticadas, pode ser utilizado para atribuir rótulos semânticos refinados a tais \textit{clusters}, ampliando a interpretabilidade dos tópicos e permitindo a construção de narrativas explicativas sobre tendências detectadas nos dados \citeonline{Meng_2024, Galli_2024}.

% No contexto deste projeto, a integração de ambos os modelos se mostra justificada. Enquanto o BERTopic viabiliza a organização automática de grandes corpora textuais provenientes da plataforma SIMCC, o GPT-4 agrega valor na etapa de rotulagem, interpretação semântica e análise contextual aprofundada. Tal combinação potencializa tanto a acurácia quanto a inteligibilidade dos resultados, conciliando rigor metodológico com clareza interpretativa. Além disso, a aplicação conjunta favorece a detecção de padrões emergentes em múltiplos idiomas, aspecto essencial dada a heterogeneidade linguística base de dados.

% Portanto, ao invés de restringir-se a abordagens puramente estatísticas ou unicamente gerativas, este trabalho adota uma perspectiva híbrida, combinando técnicas de modelagem de tópicos e de raciocínio semântico avançado, buscando suprir lacunas de interpretabilidade.
