\chapter[REFERENCIAL TEÓRICO]{REFERENCIAL TEÓRICO}
\label{chap:referencial_teorico}

Este capítulo estabelece a fundamentação teórica que sustenta o desenvolvimento do \textit{pipeline} proposto. A revisão da literatura aborda os pilares conceituais necessários para a análise de publicações científicas e para a construção do artefato de mapeamento interativo.

Iniciamos pela Ciência da Informação, contextualizando o desafio central do crescimento exponencial da produção científica e as limitações das abordagens tradicionais de recuperação. Em seguida, aprofundamos nos fundamentos técnicos do Processamento de Linguagem Natural (\gls{pln}), explorando a arquitetura dos Transformadores e o conceito de \textit{embeddings}, que são a base dos modelos modernos.

Posteriormente, detalhamos as Abordagens de Modelagem de Tópicos, comparando métodos tradicionais, como o \gls{lda}, com a arquitetura moderna do \gls{bertopic}, justificando sua escolha. Por fim, discutimos a importância da Visualização da Informação como ferramenta analítica, fundamentando a integração da ferramenta \gls{wizmap} como etapa final do artefato.

\section{Ciência da Informação e Análise de Publicações Científicas}

A explosão da produção científica global nas últimas décadas, impulsionada pela maior acessibilidade à tecnologia e pela colaboração interdisciplinar, delineia um cenário desafiador para a área da Ciência da Informação. Como destacam \citeonline{Kim_2024}, o volume crescente de publicações dificulta a atualização contínua de pesquisadores e a identificação de áreas emergentes do conhecimento. Os autores reforçam essa problemática no resumo de seu trabalho:

\begin{citacao}
A produção científica global está se expandindo exponencialmente, o que, por sua vez, exige uma melhor compreensão da ciência da ciência e, especialmente, de como as fronteiras dos campos científicos se expandem através de processos de emergência. \citeonline[Traduzido, p. 1]{Kim_2024}
\end{citacao}

Nesse contexto, estratégias tradicionais de busca baseadas em palavras-chave mostram-se limitadas, uma vez que desconsideram a complexidade semântica do léxico científico. Esse fator resulta não apenas na omissão de trabalhos relevantes, mas também na dificuldade de mapear de forma consistente o progresso em determinados campos.

Um aspecto que amplia essa complexidade é a diversidade linguística no ambiente científico. Segundo \citeonline{Xie_2020}, embora o inglês desempenhe papel predominante, uma parcela significativa da produção ocorre em outros idiomas. Metodologias convencionais de análise revelam-se insuficientes para o tratamento multilíngue, o que pode restringir a circulação global do conhecimento e reduzir a visibilidade de estudos relevantes.

\begin{citacao}
A maioria dos estudos até hoje sobre análise de tópicos tem sido baseada em publicações em língua inglesa e tem dependido fortemente da análise de evolução de tópicos baseada em citações. [...] metodologias baseadas em citações não são adequadas para analisar relações de tópicos de pesquisa multilíngues. \citeonline[Traduzido, p. 1]{Xie_2020}
\end{citacao}

Diante desse cenário, técnicas contemporâneas de \textit{Topic Modeling}, em especial aquelas fundamentadas em \textit{embeddings}, têm sido investigadas como alternativas promissoras. De acordo com \citeonline{Galli_2024}, a utilização de representações densas derivadas de modelos como o \gls{bert} potencializa a análise de grandes volumes textuais, permitindo capturar aspectos semânticos que vão além da simples coincidência lexical.

\begin{citacao}
Um componente essencial para alcançar a compreensão semântica são os \textit{embeddings} — representações numéricas que codificam o significado de palavras ou mesmo de frases — que são fundamentais no \gls{pln} para capturar relacionamentos complexos entre palavras e frases usando arquiteturas especiais conhecidas como transformadores. \citeonline[Traduzido, p. 2]{Galli_2024}
\end{citacao}

Essa capacidade favorece a identificação de padrões temáticos em documentos que não compartilham necessariamente o mesmo vocabulário. Métodos modernos, como o \gls{bertopic}, oferecem uma estrutura metodológica para a extração de tópicos a partir dessas representações vetoriais densas. A literatura aponta que a aplicação dessas ferramentas é particularmente relevante em textos científicos heterogêneos e multilíngues, como os encontrados em grandes repositórios de publicações científicas, dada a sua robustez em capturar nuances semânticas independentemente do idioma \cite{Xie_2020}.

\section{Processamento de Linguagem Natural (PLN)}

O Processamento de Linguagem Natural (\gls{pln}) é um campo multidisciplinar, situado na interseção da Inteligência Artificial, da Linguística Computacional e da Ciência da Informação. O objetivo central da área é desenvolver métodos computacionais capazes de processar, analisar, compreender e gerar a linguagem humana — seja em formato de texto ou voz — de maneira funcional \cite{Jurafsky_2009}\footnote{Refere-se à obra \textit{Speech and Language Processing}, de Daniel Jurafsky e James H. Martin. É amplamente considerado o livro-texto acadêmico padrão e a referência canônica para o ensino e estudo do Processamento de Linguagem Natural em todo o mundo.}.

Historicamente, o \gls{pln} fundamentava-se em abordagens estatísticas e regras linguísticas manuais para modelar a linguagem, conforme descrito por \cite{Manning_1999}\footnote{Refere-se à obra \textit{Foundations of Statistical Natural Language Processing} (Manning e Schütze, 1999), considerada o trabalho seminal que consolidou as abordagens estatísticas como o padrão do \gls{pln} antes da ascensão das redes neurais profundas.}. As técnicas de \gls{pln} são projetadas para extrair significado e estrutura de dados textuais, que são inerentemente não estruturados. Este processo envolve uma série de tarefas que variam desde a análise sintática (a estrutura gramatical) até a análise semântica (o significado por trás das palavras). Entre as aplicações comuns estão a classificação de textos, a tradução automática, a sumarização de documentos e a Modelagem de Tópicos (\textit{Topic Modeling}), que é o foco desta pesquisa.

A evolução recente do campo foi impulsionada pelo \textit{Deep Learning} (Aprendizado Profundo), que permitiu a criação de representações vetoriais mais precisas. Como destacam \citeonline{Galli_2024}, o \gls{pln} moderno passou a depender da capacidade de capturar o significado contextual, superando a análise baseada apenas na contagem de palavras. Essa transição para uma abordagem focada na compreensão semântica viabilizou os avanços recentes em modelagem de tópicos.

\section{A Evolução das Representações Vetoriais em PLN}

O progresso na área de \gls{pln} tem sido caracterizado pela investigação de representações vetoriais capazes de capturar não apenas a estrutura sintática, mas também os aspectos semânticos e contextuais dos textos. A evolução dessas representações partiu de abordagens estáticas para modelos dinâmicos baseados em contexto.

\subsection{Embeddings Estáticos: Limitações do Bag-of-Words}

As primeiras abordagens de sucesso, como o \textit{Word2Vec} proposto por \citeonline{Mikolov_2013}\footnote{O \textit{Word2Vec} (2013) foi seminal por introduzir duas arquiteturas eficientes, \textit{Skip-gram} e \textit{CBOW}, que aprendem vetores de palavras prevendo o contexto em que elas aparecem, baseando-se na hipótese distribucional.} e o \textit{GloVe} proposto por \citeonline{Pennington_2014}\footnote{O \textit{GloVe} (2014), ou \enquote{Global Vectors}, diferencia-se por combinar as estatísticas globais de coocorrência de palavras (como o LSA) com a modelagem baseada em janelas de contexto (como o \textit{Word2Vec}), capturando relações lineares entre palavras.}, consolidaram o conceito de \textit{embeddings}. Nesse caso, a operação algébrica subtrai o vetor de \enquote{Homem} do vetor de \enquote{Rei}, isolando o conceito de realeza, e adiciona o vetor de \enquote{Mulher}, resultando em uma representação vetorial espacialmente próxima à de \enquote{Rainha}. Isso demonstra que o modelo é capaz de codificar conceitos abstratos, como gênero, através da direção e distância entre os vetores. Estes consistem em vetores em espaços de alta dimensionalidade capazes de representar o significado aproximado de uma palavra.

A contribuição desses modelos foi permitir a quantificação do significado semântico. Em vez de tratar palavras como identificadores discretos (como em uma abordagem \textit{bag-of-words}), os \textit{embeddings} posicionam termos com significados similares próximos uns dos outros no espaço vetorial. Isso permite que relações semânticas sejam capturadas matematicamente, como no exemplo clássico \enquote{Rei - Homem + Mulher $\approx$ Rainha} \cite{Mikolov_2013}. \citeonline{Xie_2020} na literatura de \gls{pln} refere-se a este espaço vetorial como um \enquote{espaço semântico}.

O aprendizado desses vetores ocorre através do treinamento de redes neurais em tarefas de previsão de contexto, conforme ilustrado na Figura \ref{fig:cbow_skipgram}. O artigo seminal de \citeonline{Mikolov_2013} introduziram duas arquiteturas principais:

\begin{enumerate}
    \item \textbf{\gls{cbow}:} A arquitetura prevê a palavra atual (saída) com base em uma janela de palavras do contexto (entrada).
    \item \textbf{\textit{Skip-gram}:} A arquitetura inverte a lógica e usa a palavra atual (entrada) para prever as palavras do contexto (saída).
\end{enumerate}

É importante destacar que os \textit{embeddings} não são o produto final, mas sim um subproduto do treinamento: os vetores aprendidos na camada oculta da rede (\textit{PROJECTION} na figura) tornam-se a representação semântica da palavra, como indica \citeonline[p. 4]{Mikolov_2013}.

\begin{figure}[H]
    \centering
    \includegraphics[width=0.7\textwidth]{figs/Cbow_skipgram.png} 
    \caption{Arquiteturas \gls{cbow} e \textit{Skip-gram}.}
    \label{fig:cbow_skipgram}
    \legend{Fonte: \citeonline[p. 5]{Mikolov_2013}}
\end{figure}

Apesar da utilidade em capturar similaridades lexicais, esses modelos apresentavam a limitação de atribuir um único vetor fixo a cada termo, independentemente do contexto de ocorrência. Por exemplo, a palavra \enquote{banco} teria a mesma representação vetorial em \enquote{banco financeiro} e \enquote{banco da praça}. Tal restrição, usualmente referida como o problema da ambiguidade do significado da palavra (\textit{ambiguity of word meaning}), compromete a precisão em tarefas que exigem desambiguação semântica.

\subsection{A Revolução dos Transformadores e o Mecanismo de Atenção}

Uma mudança significativa no paradigma ocorreu com a introdução do modelo de Transformadores (\textit{Transformers}), proposto por \citeonline{vaswani_2017} no artigo seminal \textit{Attention Is All You Need}\footnote{Este artigo é considerado um dos trabalhos mais influentes da \gls{pln} moderna. Sua principal contribuição foi propor uma arquitetura de rede neural que dispensa totalmente as camadas recorrentes (RNN) e convolucionais, baseando-se unicamente em mecanismos de atenção para modelar dependências globais entre a entrada e a saída \cite[p. 1]{vaswani_2017}.}. Essa arquitetura diferencia-se das \gls{rnn} e convolucionais, por fundamentar-se inteiramente no mecanismo de autoatenção (\textit{self-attention}).

Por meio da autoatenção, o modelo atribui pesos diferenciados a \textit{tokens} em uma sequência, permitindo processar simultaneamente e de forma bidirecional a totalidade do contexto textual. A arquitetura do Transformador, conforme apresentado na Figura \ref{fig:transformer}, segue uma estrutura de codificador-decodificador (\textit{encoder-decoder}). O lado esquerdo do diagrama representa o Codificador, enquanto o lado direito representa o Decodificador.

\begin{figure}[H]
    \centering
    \includegraphics[width=0.4\textwidth]{figs/Transformadores.png} 
    \caption{Arquitetura do modelo Transformador.}
    \label{fig:transformer}
    \legend{Fonte: \citeonline[p. 3]{vaswani_2017}} 
\end{figure}


O \textbf{Codificador} (\textit{Encoder}) é composto por uma pilha de N camadas idênticas (no artigo original, N=6). Cada camada, por sua vez, é composta por duas subcamadas principais: um mecanismo de autoatenção \textit{multi-head} (\textit{multi-head self-attention}) e uma rede neural \textit{feed-forward} (rede neural de alimentação direta) simples e totalmente conectada. Conexões residuais seguidas de normalização de camada (\textit{Add \& Norm}) são aplicadas ao redor de cada subcamada.

O \textbf{Decodificador} (\textit{Decoder}), de forma similar, é uma pilha de N camadas. Além das duas subcamadas presentes no codificador, o decodificador insere uma terceira subcamada, que realiza a atenção \textit{multi-head} sobre a saída da pilha do codificador. Crucialmente, a subcamada de autoatenção do decodificador é \enquote{mascarada} (\textit{Masked Multi-Head Attention}). Esse mascaramento é o que garante que a previsão para uma posição \textit{i} só possa depender das saídas conhecidas em posições anteriores a \textit{i}, preservando a propriedade autorregressiva do modelo.

Embora a arquitetura completa do Transformador tenha sido projetada para tarefas de transdução de sequência (como a tradução automática), foi a sua pilha de \textbf{Codificadores} (\textit{Encoder}) que se mostrou revolucionária para tarefas de \textit{compreensão} de linguagem. A capacidade do Codificador de processar texto de forma bidirecional e gerar representações numéricas ricas em contexto estabeleceu a base para uma nova classe de modelos focados exclusivamente na representação semântica, como será detalhado a seguir.

\subsection{Embeddings Contextuais: BERT e SBERT}
\label{sec:bert_sbert}

Sobre a base arquitetônica dos Transformadores, foram desenvolvidos os modelos pré-treinados, entre os quais se destaca o \gls{bert}, introduzido por \citeonline{Devlin_2019}. O \gls{bert} utiliza a arquitetura do Codificador (\textit{Encoder}) do Transformador para gerar representações de linguagem.

A inovação fundamental do \gls{bert} foi o pré-treinamento bidirecional, que diferentemente de abordagens anteriores, como o \gls{gpt} de \citeonline{Radford_2018}, que utilizava um treinamento unidirecional (da esquerda para a direita), o \gls{bert} foi projetado para \enquote{pré-treinar representações profundamente bidirecionais, condicionando conjuntamente o contexto esquerdo e direito em todas as camadas} como apontam \citeonline[p. 1, Traduzido]{Devlin_2019}.

Para alcançar essa bidirecionalidade sem que o modelo \enquote{visse a resposta}, \citeonline{Devlin_2019} introduziu o objetivo do \gls{mlm}\footnote{O \gls{mlm} é inspirado na tarefa \textit{Cloze} \cite{Taylor_1953}, onde o modelo deve prever palavras que foram omitidas (mascaradas) de uma sentença, usando o contexto de ambas as direções (esquerda e direita) para fazer a previsão \cite[p. 1]{Devlin_2019}.}. A Figura \ref{fig:bert_vs_gpt} ilustra a diferença fundamental entre as arquiteturas de pré-treinamento, mostrando como o \gls{bert} é capaz de processar informações de toda a sequência em todas as suas camadas.

\begin{figure}[H]
    \centering
    \includegraphics[width=0.7\textwidth]{figs/Bert_vs_gpt.png} 
    \caption{Diferenças nas arquiteturas de pré-treinamento. \gls{bert} é bidirecional, \gls{gpt} é unidirecional (da esquerda para a direita) e \gls{elmo}.}
    \label{fig:bert_vs_gpt}
    \legend{Fonte: \citeonline[p. 13]{Devlin_2019}}
\end{figure}

Apesar do desempenho em tarefas de classificação, a arquitetura do original do \gls{bert} apresentou limitações para tarefas de busca de similaridade semântica ou \textit{clustering}. Conforme observador por \citeonline{Reimers_2019}, o uso do \gls{bert} \enquote{requer que ambas as sentenças sejam alimentadas na rede, o que causa um overhead computacional massivo}. Uma busca de similaridade em 10.000 sentenças, por exemplo, exigiria cerca de 50 milhões de inferências (aproximadamente 65 horas), tornando-o inviável para grandes bases de dados. Além disso, estudos empíricos demonstraram que usar os \textit{embeddings} \enquote{crus} do \gls{bert} (seja pela média das saídas ou pelo vetor do \textit{token} `[CLS]`) produz resultados insatisfatórios, muitas vezes piores do que os \textit{embeddings} estáticos como o \textit{GloVe}.

Para solucionar essa questão, \citeonline{Reimers_2019} propuseram o \gls{sbert}. Ele modifica o \gls{bert} pré-treinado, adicionando uma operação de \textit{pooling} (sendo a média, \textit{MEAN-strategy}, a mais comum) à saída do \gls{bert} para criar um \textit{embedding} de sentença de tamanho fixo.

Crucialmente, o \gls{sbert} utiliza redes siamesas\footnote{Redes siamesas são uma arquitetura onde duas ou mais redes neurais idênticas (com pesos compartilhados) processam entradas diferentes de forma independente. Elas são otimizadas para aprender uma função de similaridade, aproximando os vetores de saída para entradas similares e afastando-os para entradas diferentes.} para fazer o \textit{fine-tuning} desses \textit{embeddings} de sentença. A Figura \ref{fig:sbert_arch} ilustra a arquitetura de inferência do \gls{sbert}, onde duas sentenças (A e B) são processadas por redes \gls{bert} idênticas (com pesos compartilhados), gerando vetores de sentença \textbf{u} e \textbf{v}. Esses vetores podem, então, ser comparados eficientemente usando uma medida de similaridade, como a similaridade de cosseno (\textit{cosine-similarity}).

\begin{figure}[H]
    \centering
    \includegraphics[width=0.7\textwidth]{figs/Sbert_arch.png} 
    \caption{Arquitetura de inferência do \gls{sbert} para computar similaridade.}
    \label{fig:sbert_arch}
    \legend{Fonte: Adaptado de \citeonline[p. 3]{Reimers_2019}}
\end{figure}

\citeonline{Reimers_2019} demonstraram que essa abordagem reduz o custo computacional de encontrar o par mais similar em 10.000 sentenças de 65 horas (com \gls{bert}) para cerca de 5 segundos. Essa otimização para similaridade de sentenças permite o uso eficiente desses vetores em cenários multilíngues. A utilização de modelos pré-treinados em múltiplos idiomas (como o \textit{paraphrase-multilingual-MiniLM-L12-v2}\footnote{Disponível em: \url{https://huggingface.co/sentence-transformers/paraphrase-multilingual-MiniLM-L12-v2}.}) torna-se particularmente relevante, visto que tais modelos produzem \textit{embeddings} semanticamente consistentes mesmo em diferentes idiomas.

\subsection{Abordagens Tradicionais de Modelagem de Tópicos}

Com o crescimento exponencial de dados textuais e a consequente necessidade de organizar informação em larga escala, a modelagem de tópicos consolidou-se como uma técnica fundamental na área de \gls{pln}. Em termos gerais, trata-se de um conjunto de métodos estatísticos cujo objetivo é identificar estruturas semânticas latentes\footnote{O termo "latente" significa que os tópicos não são diretamente observáveis, mas sim inferidos estatisticamente a partir dos padrões de coocorrência de palavras no \textit{corpus}.} — denominadas \textit{tópicos} — em coleções de documentos. Assim, essas técnicas permitem inferir distribuições temáticas que não são explicitamente observáveis, mas que emergem a partir de regularidades no uso do vocabulário.

Entre as abordagens iniciais destacam-se três marcos históricos: a \gls{lsa}, a \gls{plsa} e a \gls{lda}. Esses métodos não apenas moldaram a compreensão inicial sobre a representação semântica de textos, como também estabeleceram fundamentos conceituais e metodológicos que orientaram o desenvolvimento de modelos mais avançados.

A \gls{lsa}, proposta por \citeonline{Deerwester_1990}, parte da decomposição de matrizes termo-documento por meio da técnica de \gls{svd}\footnote{A \gls{svd} é uma técnica de álgebra linear para a decomposição de matrizes que permite encontrar a melhor aproximação de uma matriz por outra de posto inferior, sendo fundamental para a redução de dimensionalidade em espaços vetoriais de termos.}. Nesse enquadramento, documentos e termos são projetados em um espaço vetorial de dimensionalidade reduzida, o que permite atenuar ruídos lexicais e capturar relações de similaridade latentes. Apesar de sua relevância histórica, a linearidade da \gls{lsa} e sua insensibilidade a variações contextuais limitam seu desempenho em cenários onde relações semânticas complexas são determinantes \cite{George_2023, Xie_2020}.

Com o intuito de superar parte dessas limitações, \citeonline{Hofmann_1999,Hofmann_2001} introduziram a \gls{plsa}, que reformulou a representação semântica a partir de um modelo probabilístico. Nessa abordagem, cada ocorrência de palavra em um documento é modelada como proveniente de um tópico latente, de forma que a probabilidade conjunta de palavra $w$ e documento $d$ é expressa como:
\[
P(w, d) = \sum_{z \in Z} P(z|d) \, P(w|z),
\]
onde $z$ representa o conjunto de tópicos latentes. Embora tenha representado um avanço em relação à \gls{lsa}, a \gls{plsa} apresenta limitações notáveis, em especial no que se refere à escalabilidade: o número de parâmetros cresce linearmente com a quantidade de documentos, o que compromete sua generalização e a torna suscetível a \textit{overfitting} \cite{Datchanamoorthy_2023}.

A evolução natural desse paradigma ocorreu com a formulação da \gls{lda}, proposta por \citeonline{Blei_2003}. Ao contrário da \gls{plsa}, a \gls{lda} incorpora uma camada Bayesiana por meio da utilização de distribuições de \textit{Dirichlet} como \textit{priors}\footnote{A \gls{lda} é um modelo generativo Bayesiano. O uso das distribuições de \textit{Dirichlet} (uma distribuição de probabilidade sobre outras distribuições) permite ao modelo tratar as misturas de tópicos nos documentos e as misturas de palavras nos tópicos como variáveis aleatórias, conferindo maior robustez e melhor generalização.}. Essa estrutura permite regularizar o modelo e definir uma distribuição de tópicos não apenas a nível de documento, mas também a nível de \textit{corpus}, resultando em maior robustez e interpretabilidade. A \gls{lda} parte da premissa de que cada documento é representado como uma mistura de tópicos, e cada tópico, por sua vez, é caracterizado por uma distribuição de palavras. Essa formulação tornou o modelo amplamente aplicável em diferentes domínios, como saúde pública \cite{Mifrah_2020} e eficiência energética \cite{Polyzos_2022}.

Apesar de sua influência, tanto a \gls{lsa} quanto a \gls{plsa} e a \gls{lda} compartilham limitações estruturais. Todas operam no paradigma de \textit{bag-of-words}\footnote{O \textit{Bag-of-Words} (Saco de Palavras) é um modelo de representação de texto que ignora a ordem e a estrutura gramatical das palavras, tratando um documento apenas como um conjunto (ou multiconjunto) de suas palavras e suas frequências.} (saco de palavras), que ignora a ordem e o contexto local das palavras. Segundo \citeonline{George_2023, Xie_2020}, isso frequentemente conduz a representações semânticas superficiais em textos técnicos ou multilíngues. \citeonline{Datchanamoorthy_2023} também reitera que a sensibilidade da \gls{lda} à definição do número de tópicos ($K$) representa um desafio adicional: valores reduzidos podem fundir tópicos distintos em um único, enquanto valores elevados podem fragmentar temas coesos em subtemas artificiais.

\begin{citacao}
A sensibilidade do \gls{lda} ao parâmetro do número de temas ($K$) é uma de suas desvantagens. Encontrar o valor ideal para ($K$) pode ser desafiador. O modelo pode simplificar excessivamente e combinar diferentes temas em um só se ($K$) for configurado muito baixo. No entanto, se ($K$) for configurado muito alto, o modelo pode se tornar muito complexo e produzir temas errôneos \cite[Tradução nossa]{Datchanamoorthy_2023}.
\end{citacao}

Essas restrições evidenciam que, embora fundamentais, tais técnicas falham em capturar o significado contextual profundo e a ordem das palavras. Essa limitação estrutural torna-os insuficientes para tarefas que exigem uma compreensão semântica robusta, especialmente em bases textuais heterogêneos ou multilíngues onde a ambiguidade lexical é alta, destacando a necessidade de abordagens que superem o paradigma \textit{bag-of-words}.

\section{BERTopic: Uma Abordagem Moderna}
\label{sec:bertopic}

As limitações das abordagens tradicionais de modelagem de tópicos, especialmente sua dependência do paradigma \textit{bag-of-words} e a falha em capturar o contexto semântico, motivaram o desenvolvimento de novos métodos. Pesquisas recentes indicam a viabilidade de tratar a modelagem de tópicos como uma tarefa de \textit{clustering} (agrupamento) de \textit{embeddings}, notavelmente nos trabalhos que introduziram o \textit{Top2Vec} \cite{Angelov_2020} e em estudos comparativos como o de \citeonline{Sia_2020}.

Nesse contexto, \citeonline{Grootendorst_2022} propôs o \gls{bertopic}, um modelo que estende a abordagem de \textit{clustering} ao introduzir uma variação do \gls{tf-idf} baseada em classes para extrair representações de tópicos. O \gls{bertopic} funciona como um \textit{pipeline} modular que consiste em três etapas principais: 1) geração de \textit{embeddings} de documentos, 2) \textit{clustering} desses \textit{embeddings} e 3) representação dos tópicos com \gls{c-tf-idf} \cite[p. 1-2]{Grootendorst_2022}.

A Figura \ref{fig:bertopic_pipeline} ilustra o fluxo geral dessa arquitetura.

\begin{figure}[H]
    \centering
    % Nota: Figura 2 do seu PDF (p. 20)
    % A fonte original é \cite{Jung_2024}
    \includegraphics[width=0.4\textwidth]{figs/Bertopic_arch.png} 
    \caption{Diagrama esquemático do \textit{pipeline} BERTopic.}
    \label{fig:bertopic_pipeline}
    \legend{Fonte: \citeonline[p. 7, Traduzido]{Jung_2024}}
\end{figure}

Na primeira etapa, \textit{Document embeddings}, os documentos são convertidos em representações vetoriais (embeddings). O \gls{bertopic} utiliza nativamente a biblioteca \gls{sbert} (\textit{Sentence-BERT}) proposta por \citeonline{Reimers_2019}, garantindo que documentos semanticamente similares sejam posicionados próximos no espaço vetorial \cite[p. 2]{Grootendorst_2022}. 

A segunda etapa, \textit{Dimension reduction}, é o \textit{reduction} desses \textit{embeddings} de alta dimensionalidade. Para que os algoritmos de \textit{reduction} funcionem de forma eficiente, é necessário primeiro combater a \enquote{maldição da dimensionalidade} (\textit{curse of dimensionality}), um fenômeno onde as distâncias entre os pontos se tornam pouco significativas em espaços com muitas dimensões \cite[p. 2]{Grootendorst_2022}. Para isso, o \gls{bertopic} emprega o \gls{umap} \cite{McInnes_2018}.

Antes de passar para as próximas etapas, precisamos contextualizar o \gls{umap}, que é uma técnica de redução de dimensionalidade que se destaca por preservar tanto a estrutura local quanto a estrutura global dos dados em um espaço de dimensão reduzida\footnote{O \gls{umap} é fundamentado em geometria Riemanniana e topologia algébrica. Ele constrói uma representação topológica dos dados em alta dimensão e busca uma representação em baixa dimensão que tenha uma estrutura topológica o mais equivalente possível \cite[p. 3-4]{McInnes_2018}.} \cite[p. 2-3]{Grootendorst_2022}. A Figura \ref{fig:umap_params} demonstra o impacto de seus dois principais hiperparâmetros.

\begin{figure}[H]
    \centering
    % Nota: Figura 3 do seu PDF (p. 21)
    % A fonte original é \cite{McInnes_2018}
    \includegraphics[width=0.7\textwidth]{figs/UMAP.png} 
    \caption{Diagrama ilustrativo do \gls{umap}, demonstrando a relação entre os hiperparâmetros \texttt{n\_neighbors} e \texttt{min\_dist} e a representação visual dos dados.}
    \label{fig:umap_params}
    \legend{Fonte: \citeonline[p. 24]{McInnes_2018}}
\end{figure}

Conforme ilustrado na Figura \ref{fig:umap_params}, o parâmetro \texttt{n\_neighbors} (número de vizinhos) controla o equilíbrio entre a preservação da estrutura global (valores altos) e local (valores baixos). O parâmetro \texttt{min\_dist} (distância mínima) ajusta a densidade dos agrupamentos, determinando a proximidade entre os pontos no espaço de baixa dimensionalidade.

Com os vetores em dimensão reduzida pelo \gls{umap}, a etapa seguinte da Figura \ref{fig:bertopic_pipeline}, \textit{Document clustering}, é o \textit{clustering} através do \gls{hdbscan} \cite{McInnes_2018}. A escolha deste método justifica-se pelas limitações dos algoritmos tradicionais de particionamento, como o \textit{K-Means}\footnote{O \textit{K-Means} é um dos algoritmos de \textit{clustering} mais populares. Ele particiona $n$ observações em $k$ agrupamentos, onde cada observação pertence ao \textit{cluster} cujo centro (média) é o mais próximo. Sua simplicidade é uma vantagem, mas ele assume \textit{clusters} de forma esférica e sensibilidade à inicialização dos centroides \cite{MacQueen_1967}.}.

O \textit{K-Means} assume que todos os agrupamentos (clusters) possuem formato esférico e densidades similares, além de forçar a inclusão de todos os pontos em algum grupo . No entanto, dados reais de publicações científicas raramente seguem esse padrão: tópicos podem ter formatos irregulares e muitos documentos podem não pertencer a nenhum tema específico (ruído).

A Figura \ref{fig:hdbscan_dataset} apresenta um cenário sintético que ilustra exatamente esse desafio de \enquote{densidade variável e ruído}, típico de dados não estruturados. Ao analisar a Figura \ref{fig:hdbscan_dataset}, observa-se a coexistência de três situações distintas no mesmo conjunto de dados:
\begin{enumerate}
    \item \textbf{Clusters de Alta Densidade:} Agrupamentos compactos (topo e esquerda), representando temas muito específicos e coesos.
    \item \textbf{Clusters de Baixa Densidade:} Agrupamentos mais dispersos (direita), representando temas mais amplos ou menos consolidados.
    \item \textbf{Ruído (\textit{Noise})} Pontos isolados espalhados pelo fundo, que não se conectam claramente a nenhum grupo.
\end{enumerate}

O HDBSCAN supera esse desafio por ser um algoritmo baseado em densidade. Diferentemente de métodos que buscam apenas a distância até um centro, o HDBSCAN identifica \enquote{ilhas} de alta densidade em um \enquote{mar} de pontos dispersos. Essa característica permite que o algoritmo:
\begin{itemize}
    \item Identifique clusters de formatos arbitrários e densidades variadas simultaneamente;
    \item Classifique pontos isolados como outliers (ruído), atribuindo-lhes o rótulo -1, em vez de força-los a integrar um tópico incoerente.
\end{itemize}

A Figura \ref{fig:hdbscan_dataset} ilustra a capacidadade de identificar agrupamentos de densidades e formas variadas, demonstrando o tipo de desafio que o algoritmo HDBSCAN é capaz de superar, como a identificação de agrupamentos de densidades e formas variadas, além de tratar outliers de forma eficiente. Essa característica é especialmente relevante em contextos de produção científica, onde coexistem tanto publicações centrais com alta densidade de tópicos quanto trabalhos periféricos ou com temas emergentes.

\begin{figure}[H]
    \centering
    % Nota: Figura 4 do seu PDF (p. 22)
    % A fonte original é \cite{Campello_2013}
    \includegraphics[width=0.4\textwidth]{figs/HDBSCAN.png} 
    \caption{Figura ilustrativa de um \textit{dataset} sintético com quatro \textit{clusters} e ruído de fundo.}
    \label{fig:hdbscan_dataset}
    \legend{Fonte: \citeonline[p. 16]{Campello_2013}}
\end{figure}

As duas etapas finais do \textit{pipeline}, é onde ocorre a geração da representação dos tópicos. Abordagens anteriores, como o \textit{Top2Vec} \cite{Angelov_2020}, baseiam-se em encontrar o centroide (o ponto médio) do \textit{cluster} e identificar as palavras mais próximas a ele. \citeonline{Grootendorst_2022} argumenta que essa abordagem é falha, pois \enquote{um \textit{cluster} nem sempre se situa dentro de uma esfera ao redor de um centroide} \cite[p. 1, Traduzido]{Grootendorst_2022}.

Para resolver o cénario dos clusters ao redor do centroide, o \gls{bertopic} introduz o \gls{c-tf-idf} (\textit{Class-based Term Frequency-Inverse Document Frequency}). A abordagem primeiro trata todos os documentos dentro de um \textit{cluster} (tópico) como um único documento concatenado. Em seguida, modifica a fórmula padrão do \gls{tf-idf} para operar a nível de classe, e não de documento.

O \gls{tf-idf} clássico é definido por \citeonline{Joachims_1997} como:
\begin{equation}
W_{t,d} = tf_{t,d} \cdot \log\left(\frac{N}{df_{t}}\right)
\label{eq:tfidf_classico}
\end{equation}
onde $W_{t,d}$ é a pontuação da palavra $t$ no documento $d$, $tf_{t,d}$ é a frequência da palavra $t$ no documento $d$, $N$ é o número total de documentos e $df_{t}$ é o número de documentos que contêm a palavra $t$.

O \gls{c-tf-idf} adapta essa lógica, onde a frequência do termo ($tf$) é calculada para a palavra $t$ dentro da classe $c$ inteira (o \textit{cluster} de documentos concatenados). A frequência inversa do documento ($idf$) é substituída pela \enquote{frequência inversa da classe}, que mede a importância da palavra $t$ em relação a todas as outras classes. A fórmula é então ajustada para:
\begin{equation}
W_{t,c} = tf_{t,c} \cdot \log\left(1 + \frac{A}{tf_{t}}\right)
\label{eq:c-tfidf}
\end{equation}
onde $tf_{t,c}$ é a frequência da palavra $t$ na classe $c$, $A$ é o número médio de palavras por classe (total de palavras dividido pelo número de classes), e $tf_{t}$ é a frequência total da palavra $t$ em todas as classes \cite[p. 3]{Grootendorst_2022}. O resultado é uma lista de palavras que destaca os termos que são mais distintivos e representativos de um tópico específico.

Embora eficaz, o \gls{c-tf-idf} pode gerar palavras-chave redundantes (ex: \enquote{modelo}, \enquote{modelagem}). O \citeonline[p. 8]{Grootendorst_2022} sugere que isso pode ser resolvido \enquote{aplicando \gls{mmr} às n palavras principais de um tópico}. O \gls{mmr}, introduzido por \citeonline{Carbonell_1998}, é uma técnica projetada especificamente para otimizar o equilíbrio entre relevância e diversidade na recuperação de informações. O algoritmo funciona de forma iterativa: ele primeiro seleciona o termo de maior relevância (maior pontuação \gls{c-tf-idf}); em seguida, para cada termo candidato subsequente, ele aplica uma penalidade com base na similaridade desse candidato com os termos já selecionados. O resultado é um conjunto de palavras-chave que não apenas representa o tema central, mas também cobre diferentes facetas semânticas desse tema, aumentando significativamente a interpretabilidade humana.

Contudo, mesmo com representações de tópicos robustas e rótulos semanticamente diversos, analisar a estrutura latente e as inter-relações de centenas de tópicos em um \textit{corpus} massivo permanece um desafio. A geração de um modelo de tópicos é apenas a primeira etapa; a descoberta de conhecimento emerge da capacidade de explorar esses resultados de forma intuitiva. Isso destaca a necessidade de técnicas que superem listas estáticas e permitam uma análise exploratória, um desafio que é central no campo da Visualização Científica e de Dados.

\section{Visualização de Dados para Análise Científica}
\label{sec:visualizacao_dados}

A geração de modelos de tópicos e \textit{embeddings}, conforme discutido nas seções anteriores, produz representações vetoriais de alta dimensionalidade que capturam a semântica do domínio. No entanto, a interpretação e o uso prático desses \textit{embeddings} representam um desafio significativo, dada a sua \enquote{opacidade, alta dimensionalidade e o grande tamanho dos conjuntos de dados modernos} \cite[p. 1, Traduzido]{Wang_2023}.

Para tornar esses vetores complexos inteligíveis, pesquisadores frequentemente aplicam técnicas de redução de dimensionalidade, como o \gls{umap} \cite{McInnes_2018} ou o \gls{t-sne} \cite{Maaten_2008}, para projetar os \textit{embeddings} em um espaço bidimensional (\textbf{2d}) ou tridimensional (\textbf{3d}). Embora essa projeção permita a visualização dos dados em um gráfico de dispersão (\textit{scatter plot}), a análise em larga escala permanece um desafio: em conjuntos de dados com milhões de pontos, \enquote{é exaustivo ou mesmo implausível inspecionar os dados ponto a ponto para entender a estrutura global} \cite[p. 2, Traduzido]{Wang_2023}.

Abordagens alternativas, como gráficos de contorno (\textit{contour plots}), podem resumir a distribuição global, mas \enquote{restringem a exploração das estruturas locais de um \textit{embedding}} \cite[p. 2, Traduzido]{Wang_2023}. Para preencher a lacuna entre a visão global (contornos) e a exploração local (pontos), ferramentas de visualização interativa tornam-se essenciais.

Neste contexto, surge o \gls{wizmap}\footnote{O repositório de código aberto do \gls{wizmap} está disponível em: \url{https://github.com/poloclub/wizmap}.}, \enquote{uma ferramenta de visualização interativa escalável que capacita pesquisadores e especialistas de domínio a explorar e interpretar \textit{embeddings} com milhões de pontos} \cite[p. 2, Traduzido]{Wang_2023}. A ferramenta emprega um \enquote{design de interação familiar semelhante a um mapa} (\textit{map-like interaction design}), permitindo que o usuário navegue pelo espaço semântico com ações de \textit{pan} e \textit{zoom}.

A interface do \gls{wizmap}, ilustrada na Figura \ref{fig:Wizmap_interface}, é dividida em três componentes principais: (A) A Visão de Mapa (\textit{Map View}), que integra as camadas de visualização; (B) O Painel de Busca (\textit{Search Panel}), que permite a filtragem por texto; e (C) O Painel de Controle (\textit{Control Panel}), para customização da visualização \cite[p. 1]{Wang_2023}.

\begin{figure}[H]
    \centering
    % Nota: Figura 1 do artigo wizmap.pdf (p. 1)
    \includegraphics[width=0.7\textwidth]{figs/Wizmap_interface.png} 
    \caption{Interface da ferramenta \gls{wizmap} e seus componentes principais.}
    \label{fig:Wizmap_interface}
    \legend{Fonte: \citeonline[p. 1]{Wang_2023}}
\end{figure}

A principal inovação do \gls{wizmap} é a sua capacidade de escalar para milhões de pontos diretamente no navegador do usuário, sem a necessidade de servidores dedicados. Isso é alcançado através do uso de tecnologias web modernas, como \gls{webgl}\footnote{Disponível em: \url{https://developer.mozilla.org/pt-BR/docs/Web/API/WebGL_API}} para renderização gráfica, \textit{Web Workers}\footnote{Disponível em: \url{https://developer.mozilla.org/pt-BR/docs/Web/API/Web_Workers_API}.} para paralelização, e a \textit{Streams API}\footnote{Disponível em: \url{https://developer.mozilla.org/pt-BR/docs/Web/API/Streams_API}.} para o carregamento de dados \cite[p. 2, 4]{Wang_2023}.

A \enquote{Visão de Mapa} (\textit{Map View}), sua interface primária, integra três camadas de visualização \cite[p. 4]{Wang_2023}:
\begin{enumerate}
    \item \textbf{Contorno de Distribuição:} Utiliza \gls{kde} para fornecer uma visão geral da estrutura global e das áreas de alta densidade.
    \item \textbf{Gráfico de Dispersão (\textit{Scatter Plot}):} Permite a investigação de \textit{embeddings} individuais em nível local.
    \item \textbf{Rótulos Multi-Resolução:} Permite uma interpretação contextual em diferentes níveis de granularidade.
\end{enumerate}

Para implementar os Rótulos Multi-Resolução, o \gls{wizmap} utiliza uma estrutura de dados \textit{quadtree}, conforme detalhado na Figura \ref{fig:wizmap_quadtree}. O \textit{quadtree} particiona recursivamente o espaço \textbf{2d} (A) em quadrantes, que são representados como nós em uma árvore (B). A ferramenta então agrega as informações de baixo para cima, permitindo que os rótulos se \enquote{ajustem em resolução à medida que os usuários aumentam o \textit{zoom}} \cite[p. 2-3]{Wang_2023}.

\begin{figure}[H]
    \centering
    % Nota: Figura 3 do artigo wizmap.pdf (p. 3)
    \includegraphics[width=0.5\textwidth]{figs/Wizmap_quadtree.png} 
    \caption{Estrutura de dados \textit{Quadtree} usada pelo \gls{wizmap} para agregação multi-resolução. (A) Particionamento recursivo do espaço \textbf{2d}. (B) Representação em árvore.}
    \label{fig:wizmap_quadtree}
    \legend{Fonte: \citeonline[p. 3]{Wang_2023}}
\end{figure}

Ferramentas como o \gls{wizmap}, que combinam redução de dimensionalidade com interfaces interativas de multi-resolução e busca semântica, são fundamentais para traduzir a saída matemática de modelos como o \gls{bertopic} em mapas de conhecimento navegáveis, facilitando a descoberta de padrões latentes em grandes base de dados textuais.