\chapter[METODOLOGIA]{METODOLOGIA}
\label{chap:metodologia}

Este estudo adota a \textit{Design Science Research} (\gls{dsr}) como sua principal estrutura metodológica para o desenvolvimento e a validação de um artefato tecnológico. A \gls{dsr} é particularmente utilizada para esta pesquisa, pois seu foco reside na criação de soluções com caráter de inovação para problemas práticos, alinhando rigor científico com relevância aplicada \cite{Dresch_2015}.

O objetivo é construir um \textit{pipeline} computacional para o mapeamento interativo de publicações científicas do Observatório de dados públicos de ciência e tecnologia da Bahia, combinando a modelagem de tópicos do \gls{bertopic} com a visualização interativa do \gls{wizmap}.

O processo de \gls{dsr} orienta o projeto de forma iterativa, desde a concepção do problema até a comunicação dos resultados, conforme ilustrado no fluxograma da Figura \ref{fig:dsr_adaptada}. Este capítulo está estruturado para detalhar cada etapa desse processo: inicia-se pela \textit{Identificação do Problema e Definição de Objetivos} (Seção 4.1), descreve o \textit{Desenvolvimento do Artefato} (Seção 4.2), detalha os procedimentos de \textit{Avaliação e Validação} (Seção 4.3) e conclui com a \textit{Apresentação dos Resultados e Comunicação} (Seção 4.4). A seguir, cada etapa do DSR é detalhada e contextualizada no escopo deste trabalho.

\begin{figure}[H]
    \centering
    % Nota: Figura 1 do artigo wizmap.pdf (p. 1)
    \includegraphics[width=0.7\textwidth]{figs/DSR.png} 
    \caption{Adaptação da Design Science Research para este projeto.}
    \label{fig:dsr_adaptada}
    \legend{Fonte: O Autor}
\end{figure}

\section{Identificação do Problema e Definição de Objetivos}

A primeira fase da \gls{dsr} (Figura \ref{fig:dsr_adaptada}) consiste na identificação de uma lacuna relevante. Atualmente, a exploração do acervo no Observatório de dados públicos de ciência e tecnologia da Bahia é limitada a abordagens de recuperação de informação baseadas em palavras-chave. Embora funcional, essa abordagem dificulta a identificação de conexões semânticas, a descoberta de temas interdisciplinares e a compreensão da estrutura global do conhecimento científico.

Essa limitação evidencia a necessidade de uma solução que vá além da busca lexical e ofereça uma exploração semântica e visual da distribuição do arcervo de pesquisas reunidas na plataforma. O problema de pesquisa, portanto, é: como um \textit{pipeline} de modelagem de tópicos (\gls{bertopic}) pode ser integrado a uma ferramenta de visualização interativa (\gls{wizmap}) para mapear o acervo de publicações do Observatório, oferecendo uma modalidade de exploração visual e semântica complementar à busca tradicional por palavras-chave, permitindo a identificação de padrões não-lineares no acervo?

A partir disso, formulam-se as \textit{Conjecturas Teóricas}: a integração do \gls{bertopic} com uma interface de mapa interativo (como o \gls{wizmap}) tem o potencial de melhorar significativamente a identificação, classificação e exploração de temas, oferecendo uma compreensão mais profunda dos dados por meio de visualizações interativas.

Para fins de escopo deste trabalho, adota-se uma definição operacional de Mapa de Conhecimento como: uma interface visual interativa que projeta um espaço vetorial de alta dimensionalidade em um plano 2D, onde a posição relativa dos documentos codifica sua afinidade semântica. O artefato não visa estabelecer relações formais (como \enquote{A é causa de B}), mas sim revelar a topologia do acervo (quais temas são vizinhos, quais são isolados), facilitando a exploração exploratória em detrimento da busca exata.

É importante delimitar que o foco desta pesquisa reside na validação da arquitetura do software e na eficácia dos algoritmos de \gls{pln} propostos. Portanto, o problema não abrange, nesta etapa do ciclo de Design Science Research, a integração em tempo real com o banco de dados de produção do Observatório, nem a realização de um estudo bibliométrico exaustivo de todo o ecossistema científico da Bahia. O problema restringe-se à prototipagem e validação do pipeline em um ambiente controlado de dados.

\section{Desenvolvimento do Artefato}
\label{sec:desenvolvimento_artefato}

Com base nos objetivos definidos na seção anterior, a etapa seguinte da \gls{dsr} é o desenvolvimento do Artefato. Neste trabalho, o artefato é um \textit{pipeline} computacional para o mapeamento interativo do conhecimento científico contido no acervo do Observatório.

Este \textit{pipeline} é projetado para executar as seguintes funções-chave:
\begin{enumerate}
    \item Realizar a modelagem de tópicos com o \gls{bertopic}, utilizando \textit{embeddings} contextuais (Seção \ref{sec:bert_sbert}) para identificar padrões temáticos de forma semântica e com alta coerência, capazes de agrupar documentos mesmo com variações lexicais.
    \item Aplicar o \gls{mmr} (\textit{Maximal Marginal Relevance}), detalhado na Seção \ref{sec:bertopic}, como uma etapa de pós-processamento para refinar os rótulos de tópicos derivados do \gls{c-tf-idf}.
    \item Apresentar os tópicos e documentos em uma interface gráfica interativa (\gls{wizmap}) para facilitar a exploração visual das conexões temáticas (Seção \ref{sec:visualizacao_dados}).
\end{enumerate}

A construção deste artefato se apoia no Estado da Técnica, detalhado no Capítulo \ref{chap:referencial_teorico}. O \textit{pipeline} integra os conceitos de Modelagem de Tópicos (\gls{bertopic}), Embeddings Contextuais (\gls{sbert}), Redução de Dimensionalidade (\gls{umap}), Algoritmos de Clusterização (\gls{hdbscan}) e Visualização de Dados Interativa (\gls{wizmap}).

\section{Estratégia de Avaliação do Artefato}
\label{sec:avaliacao_validacao}

Para aferir a qualidade e a utilidade do artefato desenvolvido, adota-se uma estratégia mista, alinhada aos conceitos de Avaliação e Validação em DSR. Neste trabalho, define-se Avaliação (\textit{Evaluation}) como a mensuração objetiva do desempenho do modelo computacional através de métricas de referência. Por sua vez, a Validação é entendida como a verificação qualitativa da coerência semântica e da aplicabilidade do artefato em representar corretamente os domínios de conhecimento, realizada através da inspeção por especialista.

\subsection{Avaliação de Desempenho (Métricas)}
\label{sec:validacao_quantitativa}

A avaliação de desempenho (Métricas) foca em avaliar objetivamente a qualidade dos tópicos gerados pelo \textit{pipeline} do \gls{bertopic}. Serão utilizadas métricas consolidadas na literatura de \gls{pln} para mensurar tanto a consistência interna quanto a distinção entre os grupos:

\begin{itemize}
    \item \textbf{Coerência de Tópicos (\textit{Topic Coherence}):} 
    A coerência avalia a consistência semântica e a interpretabilidade de um tópico. Um tópico é considerado coerente se as palavras que o representam formam um conceito semântico lógico \cite{Jung_2024}. Para esta avaliação, serão empregadas duas métricas com abordagens complementares:
    
    \textbf{1. Coerência $C_v$:} Esta métrica utiliza janelas deslizantes (\textit{sliding windows}) e similaridade de cosseno para calcular a relação contextual entre as palavras principais do tópico. O escore varia de 0 a 1, onde valores mais altos indicam maior consistência estrutural e interpretabilidade humana.
    
    \textbf{2. \gls{npmi} (\textit{Normalized Pointwise Mutual Information}):} O cálculo da \gls{npmi} foca na probabilidade estatística de coocorrência léxica. Consiste em extrair os $N$ termos mais representativos de cada tópico (ex: $N=10$) e calcular a probabilidade de sua aparição conjunta nos documentos do \textit{corpus} original, normalizada pela probabilidade de suas ocorrências individuais. A pontuação varia de +1 (coocorrência perfeita) a -1 (nunca aparecem juntas).
    
    \item \textbf{Diversidade de Tópicos (\textit{Topic Diversity}):}
    A diversidade é utilizada para garantir que o modelo não está gerando tópicos redundantes, ou seja, \textit{clusters} diferentes descritos pelas mesmas palavras-chave. Um modelo ideal deve apresentar tópicos que sejam, ao mesmo tempo, coerentes (alta pontuação de coerência) e distintos entre si (alta diversidade). Conforme \cite[p. 5, Traduzido]{Grootendorst_2022}, a métrica é calculada como a porcentagem de palavras únicas em todos os tópicos e é definida pela fórmula $M / (N \times k)$, onde $M$ é o número de palavras únicas extraídas nos $k$ termos principais de todos os $N$ tópicos \cite[p. 5]{Jung_2024}. O resultado varia de 0 (todos os tópicos são idênticos) a 1 (todos os termos em todos os tópicos são únicos).
\end{itemize}

\subsection{Validação Semântica (Inspeção por Especialista)}
\label{sec:validacao_qualitativa}

Dada a natureza não supervisionada da modelagem de tópicos, métricas quantitativas nem sempre refletem a intelegibilidade humana. Portanto, a validação foi complementada por uma análise qualitativa conduzida pelos pesquisadores responsáveis (autor e orientador), atuando como especialistas do domínio técnico.

O protocolo de inspeção adotou uma abordagem heurística baseada na verificação de \textit{Coerência Intratópico}. Foram selecionados aleatoriamente 20\% dos tópicos gerados (amostragem estratificada por densidade) para a verificação manual de dois critérios fundamentais:

\begin{enumerate}
    \item \textbf{Alinhamento de Rótulos:} Verificar se as palavras-chave geradas pelo algoritmo (c-TF-IDF/MMR) descrevem adequadamente o conteúdo dos títulos dos documentos agrupados naquele \textit{cluster}.
    \item \textbf{Detecção de Intrusos:} Identificar a presença de documentos semanticamente desconexos dentro de um agrupamento (ex: um artigo de 'Química' classificado erroneamente em um tópico de 'Direito').
\end{enumerate}

Essa etapa visa assegurar a \textit{validade interna} do artefato, garantindo que o \enquote{mapa} gerado possua lógica semântica antes de ser submetido a futuros testes de usabilidade com público externo. Adicionalmente, será conduzida uma comparação direta entre os rótulos gerados pelo método padrão (\gls{c-tf-idf}) e os refinados pelo \gls{mmr}, para aferir o ganho na interpretabilidade humana.

\section{Apresentação dos Resultados e Comunicação}
\label{sec:comunicacao}

A etapa final do ciclo \gls{dsr}, conforme o \textit{framework} de \citeonline{Dresch_2015}, é a Comunicação. Esta fase é dedicada à documentação e disseminação dos achados obtidos durante o desenvolvimento e a avaliação do artefato.

Os resultados da aplicação do \textit{pipeline} e de sua validação (conforme definido na Seção \ref{sec:avaliacao_validacao}) serão detalhados nos capítulos subsequentes. O \textit{Capítulo \ref{chap:projeto_desenvolvimento}} apresentará a implementação técnica do artefato e a arquitetura da solução. O \textit{Capítulo \ref{chap:resultados}} analisará os dados gerados, a qualidade dos tópicos (via \gls{npmi} e Diversidade) e a validade da interface de visualização.

O objetivo desta comunicação é apresentar e validar uma solução tecnológica para o problema de exploração de conhecimento no Observatório, documentando o processo de desenvolvimento e os resultados obtidos. Este trabalho visa demonstrar a aplicação prática de técnicas de modelagem de tópicos e visualização interativa, consolidando o aprendizado e a experiência adquiridos no desenvolvimento de um sistema funcional.