\chapter[Resultados Esperados]{RESULTADOS ESPERADOS}
A execução deste projeto visa a construção de um artefato computacional funcional e a geração de um conjunto de resultados que permitam uma nova forma de análise sobre o acervo de publicações da plataforma SIMCC. As expectativas estão centradas na qualidade dos produtos gerados pelo pipeline e na sua capacidade de revelar estruturas de conhecimento latentes nos dados.

Primeiramente, espera-se que a modelagem de tópicos realizada pelo BERTopic seja eficaz, resultando em agrupamentos (clusters) de publicações que sejam semanticamente coesos e representativos. A expectativa é que o pipeline consiga processar o corpus textual do SIMCC e identificar tópicos distintos, separando com clareza diferentes áreas do saber, mesmo em campos interdisciplinares onde a terminologia se sobrepõe.

Em segundo lugar, um resultado esperado crucial é a qualidade da rotulagem semântica gerada pelo GPT-4. Espera-se que os rótulos produzidos sejam significativamente mais descritivos e acurados do que uma abordagem de linha de base, como a simples extração das palavras-chave mais frequentes (c-TF-IDF). A hipótese é que o GPT-4 fornecerá títulos concisos e humanamente inteligíveis que capturem a essência temática de cada cluster, aumentando drasticamente a interpretabilidade dos resultados.

Por fim, o resultado agregado mais importante é a geração de um mapa de conhecimento funcional e explorável. O pipeline completo deverá produzir os dados necessários (vetores de documentos, atribuições de tópicos e rótulos semânticos) para alimentar a ferramenta WizMap. O artefato final permitirá a identificação de grandes áreas de concentração de pesquisa, nichos de especialização e potenciais conexões entre diferentes campos do conhecimento acadêmico na Bahia.

\section{Validação do Pipeline e seus Resultados}

Para garantir o rigor científico e a validade do artefato desenvolvido, será empregada uma abordagem de validação mista, combinando métricas quantitativas e análises qualitativas. Esta etapa é fundamental para aferir se os resultados gerados são consistentes, coerentes e úteis.

\subsection{Validação Quantitativa}

A qualidade dos tópicos gerados pelo BERTopic será avaliada objetivamente por meio de métricas consolidadas na literatura de PLN:

\begin{itemize}
    \item \textbf{Topic Coherence (Coerência de Tópicos)}: Será calculada a pontuação de coerência (e.g., C\_v ou NPMI) para os tópicos gerados. Esta métrica avalia o grau de similaridade semântica entre as palavras mais importantes de um mesmo tópico. Espera-se que o pipeline alcance altos escores de coerência, indicando que os agrupamentos são tematicamente consistentes.
    
    \item \textbf{Topic Diversity (Diversidade de Tópicos)}: Será medida a diversidade dos tópicos para garantir que o modelo não está gerando clusters redundantes. Uma alta diversidade, aliada a uma alta coerência, sugere uma modelagem de tópicos robusta e bem-sucedida.
\end{itemize}

\subsection{Validação Qualitativa}

A validação qualitativa é essencial para confirmar se os resultados quantitativos se traduzem em valor prático e interpretativo:

\begin{itemize}
    \item \textbf{Análise por Especialistas}: Uma amostra dos tópicos gerados (incluindo o rótulo do GPT-4 e os documentos mais representativos de cada cluster) será apresentada a pesquisadores ou especialistas familiarizados com o cenário acadêmico da Bahia. Eles serão solicitados a avaliar os tópicos com base em critérios como:
    \begin{itemize}
        \item \textbf{Coerência}: Os documentos dentro do cluster pertencem de fato ao mesmo tema?
        \item \textbf{Representatividade}: O rótulo gerado pelo GPT-4 descreve adequadamente o conteúdo do tópico?
        \item \textbf{Relevância}: O tópico identificado representa uma área de pesquisa reconhecível e relevante?
    \end{itemize}
    
    \item \textbf{Análise de Casos de Uso}: Serão selecionados pesquisadores ou grupos de pesquisa com linhas de atuação bem definidas e conhecidas dentro da base do SIMCC. A validação consistirá em verificar se o pipeline consegue agrupar corretamente as publicações desses pesquisadores em um cluster coeso e se o rótulo gerado pelo GPT-4 corresponde à sua linha de pesquisa. O sucesso nesta etapa fornecerá uma forte evidência da validade prática do sistema.
    
    \item \textbf{Comparação de Rótulos (Baseline vs. GPT-4)}: Para validar especificamente a contribuição do GPT-4, será conduzida uma comparação direta. Para um mesmo tópico, será apresentado o rótulo gerado pelo GPT-4 e um rótulo de linha de base (e.g., as 3 palavras-chave mais importantes do c-TF-IDF). Avaliadores humanos indicarão qual dos dois rótulos é mais informativo e preciso, permitindo quantificar a melhoria na interpretabilidade.
\end{itemize}

A combinação desses métodos de validação fornecerá uma avaliação completa e robusta do pipeline, atestando não apenas sua funcionalidade técnica, mas também sua eficácia na geração de conhecimento útil e confiável a partir dos dados da plataforma SIMCC.
