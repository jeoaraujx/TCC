% ---
% Conclusão (outro exemplo de capítulo sem numeração e presente no sumário)
% ---
\chapter[Resultados Esperados]{RESULTADOS ESPERADOS}
O estudo proposto espera alcançar uma série de resultados significativos que visam melhorar a análise e a gestão de publicações científicas na plataforma SIMCC. Em primeiro lugar, espera-se que a integração do BERTopic com o GPT-4 permita buscas mais precisas e contextuais, superando as limitações das abordagens tradicionais baseadas em palavras-chave exatas. Com essa combinação, os usuários da plataforma devem obter resultados mais relevantes e completos, especialmente em áreas interdisciplinares, onde sinônimos e variações terminológicas são frequentes.

Outro resultado esperado é a capacidade do pipeline proposto de identificar tópicos emergentes de forma mais eficiente. Ao capturar nuances semânticas que escapam das abordagens tradicionais, espera-se que os tópicos gerados sejam mais coesos e representativos, facilitando a descoberta de tendências e áreas de pesquisa em ascensão. Essa melhoria na identificação de tópicos emergentes deve contribuir para uma análise mais aprofundada e estratégica das publicações científicas.

A implementação do WizMap como ferramenta de visualização interativa também é um dos principais resultados esperados. Essa abordagem deve proporcionar uma experiência de navegação mais intuitiva e visual dos tópicos, permitindo que os usuários explorem os clusters de tópicos em diferentes níveis de granularidade. A visualização interativa deve facilitar a compreensão das relações entre os diferentes temas, tornando a análise mais acessível e enriquecedora.

Além disso, espera-se que o pipeline proposto reduza significativamente o tempo necessário para a análise de grandes volumes de publicações científicas. A automação e a otimização das etapas de modelagem de tópicos e rotulagem semântica devem tornar o processo mais eficiente e acessível aos pesquisadores, permitindo que eles se concentrem em insights e descobertas em vez de tarefas manuais e demoradas.

A rotulagem semântica realizada pelo GPT-4 também deve apresentar melhorias significativas. Espera-se que os rótulos gerados sejam mais descritivos e contextualizados, superando as limitações da rotulagem automática tradicional. Esses rótulos mais ricos e informativos devem facilitar a interpretação e a exploração dos tópicos, contribuindo para uma análise mais clara e precisa.

A validação do pipeline será realizada por meio de métricas de avaliação, como o Topic Coherence Score, que mede a coerência dos tópicos identificados. Espera-se que os tópicos gerados apresentem alta coerência semântica, demonstrando a eficácia do pipeline na análise de publicações científicas. Essa validação deve confirmar que a solução proposta é robusta e capaz de lidar com os desafios da análise de grandes volumes de dados textuais.

Por fim, espera-se que a solução proposta contribua significativamente para a gestão do conhecimento acadêmico. Ao oferecer uma ferramenta robusta e escalável para a análise e categorização de publicações científicas, a plataforma SIMCC deve ser aprimorada, proporcionando uma experiência mais eficiente e enriquecedora para os usuários. Além disso, os resultados obtidos serão documentados e publicados em eventos científicos e periódicos relevantes, contribuindo para o avanço do conhecimento na área de Ciência da Informação e Processamento de Linguagem Natural. A disseminação dos resultados deve inspirar novas pesquisas e aplicações práticas em diferentes contextos acadêmicos e industriais.

Em resumo, os resultados esperados indicam que a integração do BERTopic com o GPT-4 e a visualização interativa com o WizMap proporcionarão uma solução inovadora e eficaz para a análise de publicações científicas. Essa abordagem deve superar as limitações das abordagens tradicionais, oferecendo uma ferramenta valiosa para a gestão do conhecimento acadêmico e contribuindo para o avanço da ciência e tecnologia.