\chapter[METODOLOGIA]{METODOLOGIA}
Este estudo adota a Design Science Research (DSR) como sua principal estrutura metodológica para o desenvolvimento e a validação de um artefato tecnológico. A DSR é particularmente adequada para esta pesquisa, pois seu foco reside na criação de soluções com caráter de inovação para problemas práticos, alinhando rigor científico com relevância aplicada (\citeonline{Dresch_2015}). O objetivo é construir um pipeline computacional que combina BERTopic e GPT-4 para otimizar a análise de publicações científicas na plataforma SIMCC.

O processo de DSR orienta o projeto de forma iterativa, desde a concepção do problema até a comunicação dos resultados, conforme ilustrado no fluxograma da Figura 5. A seguir, cada etapa do DSR é detalhada e contextualizada no escopo deste trabalho.

\section{Identificação do Problema e Definição de Objetivos}
A primeira fase da DSR, representada na Figura 5 pelos campos ``Contexto'' e ``Problema'', consiste na identificação de uma lacuna relevante. Atualmente, a busca na plataforma SIMCC é limitada a palavras-chave, o que dificulta a identificação de conexões semânticas, temas interdisciplinares e publicações em diferentes idiomas. Essa limitação evidencia a necessidade de uma solução que otimize a gestão do conhecimento acadêmico. O problema de pesquisa, portanto, é: como a combinação de técnicas avançadas de modelagem de tópicos (BERTopic) e modelos de linguagem de grande escala (GPT-4) pode superar as limitações das buscas tradicionais na plataforma SIMCC, melhorando a análise e a classificação de temas emergentes?

A partir disso, formulam-se as ``Conjecturas Teóricas'': a integração do BERTopic com o GPT-4 tem o potencial de melhorar significativamente a identificação, classificação e visualização de temas emergentes, oferecendo uma compreensão mais profunda dos dados por meio de uma rotulagem semântica enriquecida e visualizações interativas.

\section{Desenvolvimento do Artefato}
Com base nos objetivos, a etapa seguinte é o desenvolvimento do ``Artefato'', que neste trabalho é o pipeline integrado ao SIMCC. Este pipeline é projetado para:

\begin{itemize}
    \item Realizar a modelagem de tópicos com o BERTopic, utilizando embeddings contextuais para identificar padrões temáticos de forma robusta.
    \item Utilizar o GPT-4 para a rotulagem semântica, gerando rótulos descritivos e precisos que facilitam a interpretação dos resultados.
    \item Apresentar os tópicos em uma interface gráfica interativa, como o WizMap, para facilitar a exploração das conexões temáticas.
\end{itemize}

A construção deste artefato se apoia no ``Estado da Técnica'' , que envolve uma revisão da literatura sobre Modelagem de Tópicos, Embeddings Contextuais, Redução de Dimensionalidade (UMAP), Algoritmos de Clusterização (HDBSCAN) e LLMs.

\section{Avaliação e Validação}
O ciclo de DSR exige uma avaliação rigorosa, representada no diagrama pelas ``Avaliações'' 1, 2 e 3. A validação do artefato (Avaliação 1: O artefato é válido?) será conduzida por meio de métricas quantitativas e qualitativas. Utilizaremos o \textit{Topic Coherence Score} para medir a coerência semântica dos tópicos gerados pelo pipeline. Adicionalmente, será realizada uma análise comparativa do tempo de análise entre o sistema atual e a solução proposta, visando demonstrar ganhos de eficiência.

A Avaliação 2 (O problema foi resolvido?) e a Avaliação 3 (As conjecturas teóricas parecem válidas?) serão respondidas ao final do experimento. A apresentação dos resultados, mostrando a precisão dos tópicos e a clareza dos rótulos gerados pelo GPT-4, servirá para validar se o artefato efetivamente soluciona o problema identificado e se as conjecturas teóricas se confirmam na prática.

\section{Apresentação dos Resultados e Comunicação}
A etapa final do ciclo DSR, embora não explicitada como uma caixa separada no diagrama, é a comunicação dos achados. Os resultados obtidos serão documentados detalhadamente neste trabalho, destacando o funcionamento do pipeline, a qualidade dos tópicos identificados e a eficácia da interface de visualização. O objetivo é não apenas apresentar uma solução tecnológica, mas também contribuir com conhecimento para as áreas de Ciência da Informação e Processamento de Linguagem Natural.

\begin{figure}[b]
    \caption{\label{dsr}Adaptação da Design Science Research para este projeto.}
    \begin{center}
        \hspace*{-2.3cm} % Ajuste de espaçamento horizontal
        \includegraphics[scale=0.5]{figs/DSRTrue.jpg}
    \end{center}
    \legend{Fonte: O Autor}
\end{figure}
