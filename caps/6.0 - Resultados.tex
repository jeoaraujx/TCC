\chapter[RESULTADOS E DISCUSSÃO]{RESULTADOS E DISCUSSÃO}
\label{chap:resultados}

Este capítulo apresenta a avaliação do artefato computacional desenvolvido, conforme a metodologia \gls{dsr} detalhada no Capítulo \ref{chap:metodologia}. Conforme definido na metodologia, a validação do artefato visa verificar se a integração do BERTopic com a ferramenta WizMap permite a identificação coerente de temas e a exploração estruturada do conhecimento, utilizando como base de teste os dados extraídos do Observatório.

A análise dos resultados está estruturada em três dimensões:
\begin{itemize}    
    \item \textbf{Resultados da Avaliação de Desempenho (Seção 6.1):} Apresenta as métricas objetivas de qualidade dos tópicos gerados, conforme planejado na Seção \ref{sec:validacao_quantitativa}, com foco na Coerência (NPMI) e Diversidade dos tópicos.
    
    \item \textbf{Validação Semântica e Discussão (Seção 6.2):} Realiza a análise semântica e interpretativa dos resultados, conforme a Seção \ref{sec:validacao_qualitativa}. Esta seção analisa os principais tópicos identificados, avalia a eficácia do refinamento de rótulos com \gls{mmr} e discute a validade do mapa de conhecimento gerado.
    
    \item \textbf{Limitações do Estudo (Seção 6.3):} Discute as limitações inerentes ao uso de um \textit{dataset} de teste (NPAI) e a natureza de protótipo do artefato, indicando caminhos para a aplicação futura no acervo completo do Observatório.
\end{itemize}

\section[Resultados da Avaliação de Desempenho]{Resultados da Avaliação de Desempenho}
\label{sec:resultado_validacao_quantitativa}

Conforme definido na metodologia (Seção \ref{sec:validacao_quantitativa}), a primeira etapa da avaliação do artefato consistiu na medição objetiva da qualidade dos tópicos gerados. Para isso, o \textit{pipeline} foi avaliado em duas métricas centrais: Diversidade de Tópicos e Coerência de Tópicos (NPMI).

\subsection[Análise de Sensibilidade e Definição de Tópicos]{Análise de Sensibilidade e Definição de Tópicos}
\label{ssec:analise_sensibilidade}

A avaliação de desempenho dos hiperparâmetros, conforme metodologia descrita na Seção \ref{sssec:clusterizacao}, resultou nos indicadores apresentados na Figura \ref{fig:analise_hiperparametro}. O gráfico relaciona o Escore de Coerência (eixo esquerdo, linha azul) e o Número de Tópicos Gerados (eixo direito, linha vermelha) em função do tamanho do cluster.

\begin{figure}[htb]
    \centering
    \includegraphics[width=0.8\textwidth]{figs/Analise_hiperparametro.png} 
    \caption{Análise de Hiperparâmetros: Coerência ($C_v$) vs. Granularidade.}
    \label{fig:analise_hiperparametro}
    \legend{Fonte: O Autor} 
\end{figure}

A análise dos dados demonstra que o valor 8 para o \textit{min\_cluster\_size} apresentou o desempenho superior no conjunto de testes. Observam-se os seguintes comportamentos:
\begin{itemize}    
    \item \textbf{Maximização da Coerência:} A curva de coerência apresenta uma trajetória ascendente a partir do valor 4, atingindo seu pico máximo ($C_v \approx 0,429$) em 8. Nota-se uma degradação acentuada da métrica a partir do valor 10, indicando que a imposição de agrupamentos maiores forçou a fusão de temas semanticamente distintos, reduzindo a qualidade interna dos tópicos.
    \item \textbf{Controle de Ruído:} O aumento do parâmetro de 6 para 8 resultou em uma redução na taxa de \textit{outliers} de 13,7\% para 8,75\%. Isso indica que o modelo conseguiu agregar documentos anteriormente dispersos em tópicos consistentes, diminuindo a perda de informação.
    \item \textbf{Granularidade:} A configuração selecionada resultou na identificação de 19 tópicos. Essa granularidade mostrou-se adequada para a análise, evitando a fragmentação excessiva observada nos valores menores (32 tópicos em tamanho 4) e a generalização excessiva dos valores maiores.
\end{itemize}

Com base nesses resultados, o modelo final foi instanciado com \textit{min\_cluster\_size=8}, garantindo robustez semântica e cobertura de dados.

\subsection[Diversidade de Tópicos]{Diversidade de Tópicos}
\label{ssec:diversidade_topicos}

A métrica de diversidade mede o quão distintos os tópicos são entre si, calculando a porcentagem de palavras únicas entre os 10 termos mais representativos de todos os tópicos gerados. Esta métrica é utilizada para avaliar se o modelo está produzindo agrupamentos redundantes.

No experimento, o modelo alcançou um índice de Diversidade de Tópicos de \textbf{0.9214} (ou 92,14\%). Este resultado indica uma baixa sobreposição de vocabulário entre os tópicos gerados. O alto índice sugere que a configuração do \textit{BERTopic}, aliada ao refinamento por \textit{MMR}, foi eficaz na segregação dos documentos em \textit{clusters} temáticos linguisticamente distintos, minimizando a redundância.

\subsection[Avaliação de Coerência e Trade-off Semântico]{Avaliação de Coerência e Trade-off Semântico}
\label{ssec:coerencia_topicos}

Para a avaliação de coerência dos tópicos, foram empregadas duas métricas com objetivos distintos: a Coerência $C_v$, focada na estabilidade estrutural, e o \gls{npmi}, focado na coocorrência léxica.

A métrica $C_v$ foi estabelecida como o indicador principal durante a otimização de hiperparâmetros (conforme Seção \ref{ssec:analise_sensibilidade}). O modelo atingiu seu ponto ótimo ($C_v \approx 0.43$) com clusters de tamanho 8, evidenciando que os agrupamentos formados possuem forte consistência interna.

Posteriormente, para avaliar o impacto da estratégia de diversificação de rótulos, comparou-se o desempenho do modelo Padrão (\textit{Standard c-TF-IDF}) contra o modelo com refinamento \gls{mmr} utilizando a métrica \gls{npmi}. A Figura \ref{fig:npmi_modelos} ilustra os resultados obtidos.

\begin{figure}[htb]
    \centering
    \includegraphics[width=0.8\textwidth]{figs/Npmi_modelos.png} 
    \caption{Comparativo de Coerência Semântica (NPMI) entre os modelos.}
    \label{fig:npmi_modelos}
    \legend{Fonte: O Autor} 
\end{figure}

A análise dos dados revela que o Modelo Padrão obteve um índice \gls{npmi} superior ($-0.0864$) em comparação ao Modelo com \gls{mmr} ($-0.2639$). Embora métricas de coerência mais altas sejam usualmente preferíveis, neste cenário específico, a divergência corrobora a eficácia do algoritmo MMR em sua função de diversificação semântica.

O Modelo Padrão tende a maximizar o \gls{npmi} ao selecionar termos de alta coocorrência imediata, frequentemente resultando em pares redundantes ou sinônimos (ex: \enquote{doença} e \enquote{vírus}). Em contrapartida, o algoritmo \gls{mmr} penaliza intencionalmente essa redundância para introduzir termos distintos (ex: \enquote{doença} e \enquote{transporte}). Como a probabilidade estatística de um termo diverso aparecer no mesmo contexto curto é menor do que a de um sinônimo, a redução no índice NPMI é um reflexo natural da maior variabilidade vocabular.

Portanto, a variação no NPMI representa um \textit{trade-off} metodológico: o modelo prioriza a informatividade e a cobertura temática em detrimento da redundância estatística.

Adicionalmente, deve-se observar que a magnitude negativa dos valores em ambos os cenários é influenciada por fatores intrínsecos ao corpus utilizado:

\begin{enumerate}
    \item \textbf{Natureza dos Dados (Textos Curtos):} O modelo foi treinado exclusivamente com \textbf{títulos} de publicações. A brevidade dos títulos reduz a probabilidade de coocorrência de termos correlatos, penalizando métricas baseadas em contagem estrita como o NPMI, que foi originalmente projetada para documentos longos.
    \item \textbf{Volume de Dados:} A validação ocorreu em um subconjunto de dados de teste, cujo volume reduzido pode limitar a estabilização de métricas estatísticas globais de coocorrência.
\end{enumerate}

Diante dessas limitações de desempenho intrínsecas, a validação semântica e discussão, apresentada a seguir, torna-se indispensável para aferir a utilidade real e a interpretabilidade dos tópicos gerados.

\section[Validação Semântica e Discussão]{Validação Semântica e Discussão}
\label{sec:resultado_validacao_qualitativa}

A validação semântica do artefato é crucial para complementar as métricas de desempenho (Seção \ref{sec:validacao_quantitativa}), focando na interpretabilidade e na utilidade prática dos tópicos identificados. Esta seção avalia a capacidade do \textit{pipeline} em gerar agrupamentos temáticos coerentes e semanticamente ricos e discute como a visualização interativa (\gls{wizmap}) contribui para a exploração do conhecimento.

\subsection[Experimento Comparativo de Representação (c-TF-IDF vs. MMR)]{Experimento Comparativo de Representação (c-TF-IDF vs. MMR)}
\label{ssec:experimento_comparativo_representacao}

A avaliação da qualidade dos rótulos baseou-se na inspeção semântica de Nuvens de Palavras (\textit{WordClouds}) pareadas, observando-se critérios de especificidade semântica e redução de redundância.

Inicialmente, a análise do Tópico 9 (Arboviroses e Saúde Pública), apresentada na Figura \ref{fig:nuvens_palavras_topico9}, ilustra o impacto do refinamento na contextualização do tema. No modelo padrão (à esquerda), a representação é dominada por termos de alta ocorrência como \enquote{dengue}, \enquote{case} (caso) e \enquote{fever} (febre), centrando a descrição apenas na doença principal e seus sintomas.
\begin{figure}[H]
    \centering
    % Use a imagem da wordcloud_topicos.jpg
    \includegraphics[width=0.8\textwidth]{figs/Topico9_saude.png} 
    \caption{Nuvens de Palavras para Arboviroses e Saúde Pública}
    \label{fig:nuvens_palavras_topico9}
    \legend{Fonte: O Autor} 
\end{figure}

Ao aplicar o MMR (à direita), observa-se que o modelo penalizou a redundância de termos genéricos, permitindo a emergência de vetores correlatos como \enquote{chikungunya} e \enquote{zika}, além de fatores ambientais e sociais como \enquote{outbreak} (surto), \enquote{transport} e \enquote{sanitation} (saneamento). O uso da diversidade transformou uma descrição genérica em uma representação que abarca o ecossistema das arboviroses.

De forma análoga, a análise do Tópico 3 (Figura \ref{fig:nuvens_palavras_topico3}) demonstra a capacidade do MMR em revelar nichos específicos que, no modelo padrão, apareciam diluídos em um tema amplo de tecnologia educacional (com termos como \enquote{aprendizagem} e \enquote{ensino}).
\begin{figure}[H]
    \centering
    % Use a imagem da wordcloud_topicos.jpg
    \includegraphics[width=0.8\textwidth]{figs/Topico3_educacao.png} 
    \caption{Nuvens de Palavras para tecnologia educacional}
    \label{fig:nuvens_palavras_topico3}
    \legend{Fonte: O Autor} 
\end{figure}

O refinamento evidenciou que o foco real deste agrupamento recai sobre aplicações para especificidades cognitivas. Termos como \enquote{schizophrenia} (esquizofrenia), \enquote{hyperactivity} (hiperatividade) e \enquote{brain} (cérebro) ganharam relevância sobre os termos genéricos, permitindo identificar a intersecção entre inteligência artificial e neurodiversidade que não estava clara na representação baseada apenas em frequência.

Por fim, o Tópico 0 (Figura \ref{fig:nuvens_palavras_topico0}) ilustra a distinção entre conceitos macroeconômicos e componentes físicos.

\begin{figure}[H]
    \centering
    % Use a imagem da wordcloud_topicos.jpg
    \includegraphics[width=0.8\textwidth]{figs/Topico0_energia.png} 
    \caption{Nuvens de Palavras para conceitos macroeconômicos e componentes físicos}
    \label{fig:nuvens_palavras_topico0}
    \legend{Fonte: O Autor} 
\end{figure}

Enquanto a abordagem c-TF-IDF priorizou termos abstratos como \enquote{market}, \enquote{price} e \enquote{fluctuation}, o MMR diversificou a representação para incluir os componentes específicos da matriz energética analisada: \enquote{ethanol}, \enquote{gasoline} e \enquote{hydrogen}. A representação diversificada oferece, portanto, uma descrição mais concreta e imediata do objeto de estudo das publicações agrupadas.

\subsection[Análise dos Tópicos Identificados]{Análise dos Tópicos Identificados}
\label{ssec:analise_topicos}

Para superar a limitação de uma análise puramente descritiva, foi conduzida uma auditoria qualitativa nos tópicos identificados pelo pipeline. Os resultados principais, contendo os tópicos mais populosos e suas palavras-chave representativas, estão sumarizados na Tabela 2. Adicionalmente, a Figura 14 apresenta as nuvens de palavras dos agrupamentos selecionados, permitindo a inspeção visual da coesão semântica.
\begin{table}[H]
    \centering
    \caption{Tópicos Mais Populosos e suas Palavras-Chave Representativas.}
    \label{tab:topicos_principais}
    \begin{tabularx}{\textwidth}{|l|X|r|}
        \toprule
        \textbf{ID Tópico} & \textbf{Palavras-Chave} & \textbf{Nº Documentos} \\
        \midrule
        \textbf{0} & brazil, market, ethanol, gasoline, rate & 31 \\
        \textbf{1} & inovacao, empreendedorismo, pesquisa, aplicacoes, intelectual & 30 \\
        \textbf{2} & soil, fractal, oil, microrrelieve, detection & 27 \\
        \textbf{3} & aprendizagem, virtual, blind, inteligencia, artificial & 26 \\
        \textbf{4} & conhecimento, academico, validacao, gerencia, decision & 26 \\
        \textbf{5} & computador, sistema, computacao, conhecimento, organizacoes & 24 \\
        \textbf{6} & network, synchronization, semantic, sincronizacao, protein & 17 \\
        \textbf{7} & robotica, escola, makers, facial, protetor & 16 \\
        \textbf{8} & pain, chronic, fibromyalgia, postural, vibration & 13 \\
        \textbf{9} & dengue, fever, correlation, chikungunya, transport & 12 \\
        \bottomrule
    \end{tabularx}
    \legend{Fonte: O Autor}
\end{table}

\begin{figure}[H]
    \centering
    % Use a imagem da wordcloud_topicos.jpg
    \includegraphics[width=1.0\textwidth]{figs/wordcloud_topicos.png} 
    \caption{Nuvens de Palavras para Tópicos Selecionados.}
    \label{fig:nuvens_palavras}
    \legend{Fonte: O Autor} 
\end{figure}

Como critério de validação externa, os tópicos gerados foram confrontados com as Grandes Áreas do Conhecimento do CNPq. A análise revelou que os agrupamentos semânticos refletem espontaneamente as divisões disciplinares formais, conforme a distribuição observada na amostra auditada:
\begin{itemize}
    \item \textbf{Ciências da Saúde:} Representada por tópicos de alta coesão como o Tópico 8 (Dor/Fibromialgia), Tópico 9 (Arboviroses/Dengue) e Tópico 16 (Pandemia/COVID-19), que alinham-se diretamente com as subáreas da Medicina e Saúde Coletiva.
    \item \textbf{Ciências Exatas e da Terra:} Identificou-se forte presença de temas de Física e Engenharias, exemplificados pelo Tópico 0 (Energia/Combustíveis), Tópico 12 (Engenharia de Software) e Tópico 13 (Física/Fractais e Wavelets).
    \item \textbf{Ciências Humanas e Sociais:} A modelagem capturou nuances de áreas como Educação (Tópico 3 - Aprendizagem/IA), Sociologia Política (Tópico 18 - Fake News/Participação) e Turismo (Tópico 17 - Mobilidade Urbana e Turismo).
\end{itemize}

Este alinhamento entre os clusters matemáticos e a taxonomia oficial sugere que o modelo foi capaz de capturar a estrutura disciplinar subjacente ao acervo sem supervisão prévia.

Para aferir a consistência global, os tópicos foram classificados quanto à sua qualidade interpretativa. Na amostra dos tópicos mais representativos, estima-se que acima de 85\% apresentaram Alta Especificidade, definindo um tema central claro e sem ambiguidade.

No entanto, é necessário reportar a existência de agrupamentos de menor qualidade, que constituem uma limitação do ajuste não supervisionado:
\begin{enumerate}
    \item \textbf{Tópicos Genéricos/Mistos:} O Tópico 10, por exemplo, apresentou uma mistura de termos metodológicos (\enquote{bibliografico}, \enquote{colaborativo}) com temas desconexos (\enquote{cancer}, \enquote{dengue}). Isso indica um agrupamento baseado mais na estrutura de escrita científica (termos comuns em resumos) do que em um domínio de conhecimento específico.
    \item \textbf{Ruído (Outliers):} O grupo de ruído (Tópico -1), gerado nativamente pelo algoritmo HDBSCAN, conteve documentos que misturavam termos sem relação semântica aparente, como \enquote{espiritualidade}, \enquote{chagas} e \enquote{judicial}. A segregação eficaz desses documentos neste cluster de descarte valida a capacidade do modelo de limpar os tópicos principais, evitando que documentos desconexos contaminem os clusters de alta qualidade.
\end{enumerate}

Essa auditoria confirma que, embora o modelo tenha sucesso em identificar os grandes domínios científicos, a presença residual de tópicos genéricos (como o Tópico 10) reforça a necessidade de curadoria humana ou refinamento de stopwords para a aplicação bibliométrica definitiva.

\subsection[Discussão do Mapa de Conhecimento Interativo]{Discussão do Mapa de Conhecimento Interativo}
\label{ssec:mapa_conhecimento}

O artefato final, visualizado através da ferramenta WizMap na Figura \ref{fig:wizmap_geral}, atua como uma interface de navegação sobre o espaço semântico calculado. Embora não constitua uma ontologia formal, a projeção atua funcionalmente como um mapa de conhecimento ao permitir que o pesquisador identifique visualmente as fronteiras e intersecções entre os domínios científicos. As áreas de maior densidade (\textit{clusters}) representam a consolidação de tópicos, enquanto a distância euclidiana entre os pontos serve como proxy para a dissimilariedade temática.

% --- IMAGEM DO WIZMAP ---
\begin{figure}[H]
    \centering
    % Use a imagem do WizMap que você forneceu (image_2f68cf.png)
    \includegraphics[width=1.0\textwidth]{figs/wizmap_overview.png} 
    \caption{Visão Geral do Mapa de Conhecimento Interativo (WizMap).}
    \label{fig:wizmap_geral}
    \legend{Fonte: O Autor} 
\end{figure}

Na Figura \ref{fig:wizmap_geral}, as áreas de maior densidade representam os \textit{clusters} de publicações (tópicos). Os rótulos ilustram os temas centrais desses agrupamentos. A análise visual da projeção 2D aponta os seguintes padrões espaciais:

\begin{itemize}
    \item \textbf{Proximidade Espacial:} Tópicos com termos relacionados aparecem, em alguns casos, espacialmente próximos. Observa-se, por exemplo, a adjacência entre o tópico \enquote{dengue-fever-chikungunya-correlation} e o tópico \enquote{rainfall-amazonia-infection-correlation}. Esta justaposição sugere uma relação contextual nos dados entre arboviroses e fatores ambientais/regionais.

    \item \textbf{Agrupamentos Temáticos:} Algumas áreas do mapa exibem uma concentração de múltiplos tópicos. Nota-se um agrupamento que contém temas como \enquote{growth-protein-pathway-chitin} e \enquote{wavelet-fractal-conductor-dimension}, sugerindo uma intersecção, no \textit{dataset}, de pesquisas em biologia molecular com processamento de sinais ou física.

    \item \textbf{Distanciamento Temático:} Tópicos com semântica dissimilar, como \enquote{fake-news-politico-participacao} e \enquote{educacional-escola-robotico-robotica}, aparecem em regiões espacialmente distantes no mapa, indicando a separação de domínios de pesquisa.

    \item \textbf{Funcionalidade de Exploração:} A interface do \gls{wizmap} permite a exploração dinâmica dos dados. O usuário pode aplicar \textit{zoom} para inspecionar \textit{clusters} densos, ou passar o mouse sobre pontos individuais para visualizar seus \textit{tooltips} (contendo o título e o tópico, conforme definido na Seção \ref{ssec:exportacao_wizmap}). Esta funcionalidade de navegação oferece um método de exploração dos temas e suas relações espaciais, que difere da análise de tabelas ou listas estáticas.
\end{itemize}

O mapa, portanto, funciona como uma interface navegável para os dados gerados pelo \textit{pipeline}. Ele apresenta a organização dos tópicos e sua distribuição espacial, permitindo ao usuário explorar o \textit{dataset} e as relações temáticas de forma visual.

\subsection[Comparativo Qualitativo: Busca Lexical e Exploração Semântica]{Comparativo Qualitativo: Busca Lexical e Exploração Semântica}
\label{ssec:busca_lexical_exploracao}

Para estabelecer uma linha de base de comparação, foi realizado um procedimento de recuperação de informação no sistema atual do Observatório utilizando o descritor \enquote{Dengue}. A Figura \ref{fig:busca_lexical} apresenta o resultado retornado pela plataforma.

\begin{figure}[H]
    \centering
    % Use a imagem do WizMap que você forneceu (image_2f68cf.png)
    \includegraphics[width=0.7\textwidth]{figs/Observatorio_busca_lexical.png} 
    \caption{Resultado da busca lexical pelo termo \enquote{Dengue} no Observatório.}
    \label{fig:busca_lexical}
    \legend{Fonte: O Autor} 
\end{figure}

Conforme observado na Figura \ref{fig:busca_lexical}, o mecanismo de busca opera sob um paradigma estritamente lexical. O algoritmo filtra e exibe apenas os registros em que a cadeia de caracteres \enquote{Dengue} (ou variações morfológicas próximas) consta explicitamente no título da publicação. Nota-se que a dependência da correspondência exata do termo pode, inclusive, gerar ruído na recuperação, trazendo resultados semanticamente distintos do objetivo da pesquisa, como observado no terceiro item da lista, recuperado devido à similaridade gráfica com o termo de busca.

Essa restrição terminológica resulta em um problema de recuperação conhecido como \enquote{silêncio} ou falsos negativos. Publicações relevantes que abordam o domínio das arboviroses, controle vetorial ou vetores (ex: \enquote{Aedes aegypti}, \enquote{Zika}, \enquote{Chikungunya}) mas que não contêm o token específico \enquote{Dengue} no título, são excluídas da lista de resultados.

\begin{figure}[H]
    \centering
    % Use a imagem do WizMap que você forneceu (image_2f68cf.png)
    \includegraphics[width=0.7\textwidth]{figs/Cluster_wizmap.png} 
    \caption{Cluster de correlação de termos relacionados a arboviroses no mapa de conhecimento interativo.}
    \label{fig:cluster_arboviroses}
    \legend{Fonte: O Autor} 
\end{figure}

Em contraste, ao analisar o Tópico 4 gerado pelo pipeline proposto (conforme Tabela 2) e a figura \ref{fig:cluster_arboviroses}, verifica-se o agrupamento de termos como \enquote{febre} e \enquote{chikungunya} no mesmo cluster de \enquote{dengue}. Isso indica que o modelo vetorial foi capaz de conectar documentos pelo contexto semântico compartilhado e mitigar o silêncio na recuperação identificado no sistema atual.

No que tange à validação do objetivo central de \enquote{melhorar a capacidade de exploração}, os resultados da inspeção visual (demonstrados na Seção \ref{ssec:analise_topicos}) demonstram que o artefato é eficaz em agrupar documentos por afinidade semântica, uma funcionalidade inexistente na busca lexical atual (demonstrada na Seção \ref{ssec:busca_lexical_exploracao}). A validação por especialista confirma que os agrupamentos (clusters) correspondem a domínios reais de conhecimento. Portanto, considera-se que o artefato resolve o problema da ausência de ferramentas de navegação semântica, validando sua utilidade estrutural como um complemento à busca tradicional, embora a mensuração de ganhos de eficiência (tempo de busca por usuário final) permaneça como uma proposta para trabalhos futuros.

\section[Limitações do Estudo]{Limitações do Estudo}
\label{sec:limitacoes}

A interpretação dos resultados deve considerar as limitações inerentes ao escopo experimental e à natureza dos dados utilizados.

\begin{itemize}
    \item \textbf{Conjunto de Dados (Prototipagem vs. Produção):} Uma característica deste ciclo de pesquisa foi a utilização de um subconjunto de dados estático (NPAI) em detrimento da conexão direta com o acervo completo do Observatório. Embora isso impeça a generalização estatística dos tópicos encontrados para toda a ciência baiana (validade externa bibliométrica), essa restrição foi necessária para viabilizar a validação técnica do artefato (validade interna). O uso de um ambiente controlado permitiu o ajuste fino dos algoritmos sem as latências e restrições de segurança inerentes ao acesso a bancos de dados de produção governamentais. Portanto, os resultados validam a tecnologia desenvolvida, cumprindo o objetivo de engenharia proposto, deixando a aplicação em larga escala como uma etapa subsequente de implantação.

    \item \textbf{Dependência dos Hiperparâmetros:} A configuração do \textit{pipeline}, especialmente os hiperparâmetros de modelagem (Seções \ref{sssec:reducao_dimensionalidade} e \ref{sssec:clusterizacao}), foi diretamente influenciada pelo volume reduzido do \textit{dataset} de teste. Parâmetros como \texttt{n\_neighbors} (UMAP) e \texttt{min\_cluster\_size} (HDBSCAN) foram ajustados para permitir a descoberta de tópicos de nicho em um \textit{corpus} pequeno. Estes parâmetros não seriam diretamente transferíveis para o acervo completo do Observatório, que, sendo ordens de magnitude maior, exigiria uma nova etapa de \textit{tuning} (ajuste fino).

    \item \textbf{Natureza do Texto de Entrada:} O \textit{pipeline} foi treinado utilizando exclusivamente os \textbf{títulos} das publicações. Textos curtos, por definição, oferecem baixa co-ocorrência de palavras. Este fator contextual é a explicação técnica mais provável para o escore negativo da métrica \gls{npmi} (Seção \ref{ssec:coerencia_topicos}), que é uma métrica estatística dependente de co-ocorrência. Os resultados de coerência poderiam ser diferentes se o modelo fosse treinado em textos mais longos, como os resumos (\textit{abstracts}).

    \item \textbf{Limitação da Métrica de Coerência:} Conforme discutido na Seção \ref{ssec:coerencia_topicos}, a própria métrica \gls{npmi} apresenta uma limitação contextual. Ela foi projetada para avaliar modelos estatísticos (como o \gls{lda}) e pode não ser a ferramenta de avaliação de desempenho ideal para modelos baseados em similaridade semântica (como o \gls{bertopic}). A Validação Semântica e Discussão (Seção \ref{sec:validacao_qualitativa}) foi, portanto, necessária para avaliar a interpretabilidade dos tópicos.

    \item \textbf{Natureza do Artefato (DSR):} O artefato desenvolvido é um \textbf{protótipo} executado em um ambiente de desenvolvimento (Google Colab). Ele não está integrado ao \textit{pipeline} de produção, ao banco de dados dinâmico ou à interface de usuário existente do Observatório (descritos na Seção \ref{ssec:base_observatorio}). A sua função neste experimento é demonstrar a \textit{viabilidade} da metodologia, e não apresentar uma solução de \textit{software} em produção.
   
    \item \textbf{Rigidez na Hierarquia de Visualização (WizMap):} Embora a ferramenta WizMap seja eficiente para a renderização escalável de embeddings, observou-se uma limitação na integração de seus mecanismos de resumo automático com os tópicos pré-calculados pelo BERTopic. O WizMap utiliza algoritmos internos baseados em frequência de termos para gerar os rótulos dinâmicos nos níveis de zoom mais afastados (visão macro). Esse comportamento pode sobrepor a classificação semântica refinada do BERTopic, resultando, por vezes, na exibição de termos de alta frequência mas baixa relevância semântica (como nomes de autores ou termos institucionais comuns no corpus) em detrimento dos descritores temáticos. Essa característica exigiria uma etapa adicional de limpeza agressiva dos metadados de entrada especificamente para a visualização, ou o desenvolvimento de uma camada de visualização customizada que respeitasse estritamente os rótulos do c-TF-IDF em todas as escalas de resolução.
\end{itemize}