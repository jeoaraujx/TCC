\chapter[RESULTADOS E DISCUSSÃO]{RESULTADOS E DISCUSSÃO}
\label{chap:resultados}

Este capítulo apresenta a avaliação do artefato computacional desenvolvido, conforme a metodologia \gls{dsr} detalhada no Capítulo \ref{chap:metodologia}. Após a descrição da arquitetura e implementação do \textit{pipeline} no Capítulo \ref{chap:projeto_desenvolvimento}, esta seção tem como objetivo central executar a etapa de \enquote{Avaliação e Validação} .

A avaliação visa responder às conjecturas teóricas e ao problema de pesquisa: a integração do \gls{bertopic} com uma ferramenta de visualização interativa (\gls{wizmap}) de fato melhora a exploração e a descoberta de temas emergentes no acervo de publicações?

Para conduzir esta análise, o capítulo está estruturado da seguinte forma:
\begin{itemize}    
    \item \textbf{Validação Quantitativa (Seção 6.1):} Apresenta as métricas objetivas de qualidade dos tópicos gerados, conforme planejado na Seção \ref{sec:validacao_quantitativa}, com foco na Coerência (NPMI) e Diversidade dos tópicos.
    
    \item \textbf{Validação Qualitativa (Seção 6.2):} Realiza a análise semântica e interpretativa dos resultados, conforme a Seção \ref{sec:validacao_qualitativa}. Esta seção analisa os principais tópicos identificados, avalia a eficácia do refinamento de rótulos com \gls{mmr} e discute a validade do mapa de conhecimento gerado.
    
    \item \textbf{Limitações do Estudo (Seção 6.3):} Discute as limitações inerentes ao uso de um \textit{dataset} de teste (NPAI) e a natureza de protótipo do artefato, indicando caminhos para a aplicação futura no acervo completo do Observatório.
\end{itemize}

\section[Validação Quantitativa]{Validação Quantitativa}
\label{sec:resultado_validacao_quantitativa}

Conforme definido na metodologia (Seção \ref{sec:validacao_quantitativa}), a primeira etapa da avaliação do artefato consistiu na medição objetiva da qualidade dos tópicos gerados. Para isso, o \textit{pipeline} foi avaliado em duas métricas centrais: Diversidade de Tópicos e Coerência de Tópicos (NPMI).

\subsection[Diversidade de Tópicos]{Diversidade de Tópicos}
\label{ssec:diversidade_topicos}

A diversidade de tópicos mede o quão distintos os tópicos são entre si, calculando a porcentagem de palavras únicas entre os 10 termos mais representativos de todos os tópicos gerados. Esta métrica é utilizada para avaliar se o modelo está produzindo agrupamentos redundantes.

No experimento, o modelo alcançou um índice de Diversidade de Tópicos de \textbf{0.9214} (ou 92,14\%).

Este resultado indica a proporção de palavras-chave únicas que compõem os rótulos dos tópicos. Um índice nesta magnitude sugere uma baixa sobreposição lexical entre as representações textuais de cada tópico. Tal fato indica que a combinação de parâmetros do \gls{bertopic} (Seção \ref{ssec:modelagem_topicos}) resultou em \textit{clusters} com representações de palavras-chave distintas entre si.

\subsection[Coerência de Tópicos (NPMI)]{Coerência de Tópicos (NPMI)}
\label{ssec:coerencia_topicos}

A Coerência de Tópicos, calculada pela métrica \gls{npmi} (\textit{Normalized Pointwise Mutual Information}), avalia a interpretabilidade de um tópico medindo a frequência com que suas palavras-chave mais representativas co-ocorrem (aparecem juntas) no \textit{corpus} de texto original. A Figura \ref{fig:npmi_por_topico} ilustra a pontuação de cada tópico individualmente.

% Figura que você forneceu
\begin{figure}[htb]
    \centering
    \includegraphics[width=1.0\textwidth]{figs/npmi_por_topico.png} 
    \caption{Gráfico de Coerência NPMI por Tópico.}
    \label{fig:npmi_por_topico}
    \legend{Fonte: O Autor} 
\end{figure}

A pontuação média de \gls{npmi} em todos os tópicos foi de \textbf{-0.2095}.

Um escore de \gls{npmi} negativo, como o obtido, indica que as palavras-chave que definem os tópicos raramente co-ocorrem nos documentos originais. Este resultado é \textbf{metodologicamente esperado} e deve ser interpretado à luz de três fatores contextuais deste projeto:

\begin{enumerate}
    \item \textbf{Natureza dos Dados (Textos Curtos):} O \textit{pipeline} foi treinado exclusivamente em \textbf{títulos} de publicações. Títulos são, por natureza, textos muito curtos (baixa co-ocorrência). A métrica \gls{npmi} foi projetada para documentos longos (resumos ou textos completos), onde palavras de um mesmo conceito (ex: \enquote{rede} e \enquote{neural}) têm espaço para aparecer juntas.
    
    \item \textbf{Volume dos Dados (Dataset de Teste):} O modelo foi treinado em um \textit{dataset} de teste (NPAI) de volume reduzido. Métricas de coerência estatística como a \gls{npmi} necessitam de um \textit{corpus} massivo para encontrar padrões de co-ocorrência significativos.
    
    \item \textbf{Metodologia (BERTopic vs. LDA):} O \gls{bertopic} é um modelo que agrupa por \textbf{similaridade semântica} (via \textit{embeddings}) e não por co-ocorrência estatística de palavras (como o \gls{lda}). É documentado na literatura (Seção \ref{chap:trabalhos_correlatos}) que modelos baseados em \textit{embeddings} frequentemente apresentam escores \gls{npmi} mais baixos que o \gls{lda}, mesmo que seus tópicos sejam qualitativamente mais lógicos e úteis para a interpretação humana.
\end{enumerate}

Considerando o contexto deste estudo de caso, o índice de diversidade de 92,14\% indica uma baixa sobreposição de palavras entre os tópicos. Por outro lado, o escore de coerência \gls{npmi} negativo (-0.2095) sugere que as palavras-chave de um mesmo tópico raramente co-ocorrem nos textos originais (títulos). A literatura aponta que a métrica \gls{npmi} é dependente de co-ocorrência, um padrão menos frequente em textos curtos e em modelos baseados em similaridade semântica (como o \gls{bertopic}), em oposição a modelos estatísticos (como o \gls{lda}). Diante disso, a avaliação do artefato prossegue com a Validação Qualitativa, que se concentra na análise da interpretabilidade e na utilidade prática dos tópicos gerados.

\section[Validação Qualitativa]{Validação Qualitativa}
\label{sec:resultado_validacao_qualitativa}

A validação qualitativa do artefato é crucial para complementar as métricas quantitativas (Capítulo \ref{sec:validacao_quantitativa}), focando na interpretabilidade e na utilidade prática dos tópicos identificados. Esta seção avalia a capacidade do \textit{pipeline} em gerar agrupamentos temáticos coerentes e semanticamente ricos e discute como a visualização interativa (\gls{wizmap}) contribui para a exploração do conhecimento.

\subsection[Análise dos Tópicos Identificados]{Análise dos Tópicos Identificados}
\label{ssec:analise_topicos}

A inspeção manual dos tópicos, utilizando as palavras-chave mais representativas geradas pelo \gls{mmr} (Seção \ref{ssec:refinamento_rotulos}), revelou a formação de agrupamentos temáticos distintos. O modelo identificou um total de \textbf{42 tópicos} no \textit{dataset} de teste do NPAI (excluindo o tópico -1, que representa ruído).

A Tabela \ref{tab:topicos_principais} apresenta os 10 maiores tópicos identificados, ordenados pela quantidade de documentos, junto com suas 5 palavras-chave mais representativas.

% --- TABELA (Seu texto está ótimo) ---
\begin{table}[H]
    \centering
    \caption{Tópicos Mais Populosos e suas Palavras-Chave Representativas.}
    \label{tab:topicos_principais}
    \begin{tabularx}{\textwidth}{|l|X|r|}
        \toprule
        \textbf{ID Tópico} & \textbf{Palavras-Chave} & \textbf{Nº Documentos} \\
        \midrule
        \textbf{0} & fibromialgia, dor, cronico, postural, afetivo & 22 \\
        \textbf{1} & arquitetura, academico, educacao, professor, fundamental & 13 \\
        \textbf{2} & robotica, educacional, escolar, robotico, crianca & 12 \\
        \textbf{3} & design, inovador, criativo, processo, empreendedor & 12 \\
        \textbf{4} & dengue, febre, chikungunya, zika, infeccao & 12 \\
        \textbf{5} & software, programacao, sistema, desenvolvimento, gestao & 11 \\
        \textbf{6} & amazonia, floresta, chuva, infeccao, ambiental & 10 \\
        \textbf{7} & wavelet, fractal, condutor, dimensao, sinal & 9 \\
        \textbf{8} & cancer, imune, social, colaborativo, saude & 9 \\
        \textbf{9} & ontologia, framework, desenvolvimento, sistemas, arquitetura & 9 \\
        \bottomrule
    \end{tabularx}
    \legend{Fonte: O Autor}
\end{table}

A análise da Tabela \ref{tab:topicos_principais} e da Figura \ref{fig:nuvens_palavras} (que ilustra as nuvens de palavras para os tópicos mais relevantes) permite observar a capacidade do \gls{bertopic} em extrair temas emergentes e bem definidos, mesmo em um \textit{corpus} desafiador de títulos curtos.

% --- NUVENS DE PALAVRAS (Caminho da figura corrigido) ---
\begin{figure}[H]
    \centering
    % Use a imagem da wordcloud_topicos.jpg
    \includegraphics[width=1.0\textwidth]{figs/wordcloud_topicos.png} 
    \caption{Nuvens de Palavras para Tópicos Selecionados.}
    \label{fig:nuvens_palavras}
    \legend{Fonte: O Autor} 
\end{figure}

Observa-se, por exemplo:
\begin{itemize}
    \item \textbf{Tópico 0 (Fibromialgia e Dor Crônica):} Definido por termos médicos, indicando uma área de pesquisa focada em saúde e condições crônicas.
    \item \textbf{Tópico 1 (Arquitetura e Educação):} Agrupa pesquisas sobre o ensino e o ambiente acadêmico, sugerindo um nicho de estudos pedagógicos na área.
    \item \textbf{Tópico 2 (Robótica Educacional):} Evidencia um subcampo de aplicação da robótica voltado para o contexto escolar e infantil.
    \item \textbf{Tópico 4 (Dengue e Arboviroses):} Um tópico de saúde pública, indicando pesquisas sobre doenças transmitidas por vetores. Nota-se a presença de termos relacionados como \enquote{dengue}, \enquote{febre}, \enquote{chikungunya} e \enquote{infecção}.
\end{itemize}
Esses exemplos ilustram que, apesar do escore \gls{npmi} médio negativo (discutido na Seção \ref{ssec:coerencia_topicos}), os rótulos gerados com o auxílio do \gls{mmr} apresentam coerência qualitativa. A aplicação do \gls{mmr} busca reduzir a redundância, apresentando palavras-chave que cobrem diferentes aspectos de um mesmo tópico.

\subsection[Discussão do Mapa de Conhecimento Interativo]{Discussão do Mapa de Conhecimento Interativo}
\label{ssec:mapa_conhecimento}

O artefato final do \textit{pipeline} é um mapa de conhecimento interativo, gerado pela ferramenta \gls{wizmap} (Figura \ref{fig:wizmap_geral}). Esta visualização posiciona cada publicação (representada por um ponto) no espaço bidimensional calculado pelo \gls{umap} (Seção \ref{sssec:reducao_dimensionalidade}).

% --- IMAGEM DO WIZMAP ---
\begin{figure}[H]
    \centering
    % Use a imagem do WizMap que você forneceu (image_2f68cf.png)
    \includegraphics[width=1.0\textwidth]{figs/wizmap_overview.png} 
    \caption{Visão Geral do Mapa de Conhecimento Interativo (WizMap).}
    \label{fig:wizmap_geral}
    \legend{Fonte: O Autor} 
\end{figure}

Na Figura \ref{fig:wizmap_geral}, as áreas de maior densidade representam os \textit{clusters} de publicações (tópicos). Os rótulos ilustram os temas centrais desses agrupamentos. A análise visual da projeção 2D aponta os seguintes padrões espaciais:

\begin{itemize}
    \item \textbf{Proximidade Espacial:} Tópicos com termos relacionados aparecem, em alguns casos, espacialmente próximos. Observa-se, por exemplo, a adjacência entre o tópico \enquote{dengue-fever-chikungunya-correlation} e o tópico \enquote{rainfall-amazonia-infection-correlation}. Esta justaposição sugere uma relação contextual nos dados entre arboviroses e fatores ambientais/regionais.

    \item \textbf{Agrupamentos Temáticos:} Algumas áreas do mapa exibem uma concentração de múltiplos tópicos. Nota-se um agrupamento que contém temas como \enquote{growth-protein-pathway-chitin} e \enquote{wavelet-fractal-conductor-dimension}, sugerindo uma intersecção, no \textit{dataset}, de pesquisas em biologia molecular com processamento de sinais ou física.

    \item \textbf{Distanciamento Temático:} Tópicos com semântica dissimilar, como \enquote{fake-news-politico-participacao} e \enquote{educacional-escola-robotico-robotica}, aparecem em regiões espacialmente distantes no mapa, indicando a separação de domínios de pesquisa.

    \item \textbf{Funcionalidade de Exploração:} A interface do \gls{wizmap} permite a exploração dinâmica dos dados. O usuário pode aplicar \textit{zoom} para inspecionar \textit{clusters} densos, ou passar o mouse sobre pontos individuais para visualizar seus \textit{tooltips} (contendo o título e o tópico, conforme definido na Seção \ref{ssec:exportacao_wizmap}). Esta funcionalidade de navegação oferece um método de exploração dos temas e suas relações espaciais, que difere da análise de tabelas ou listas estáticas.
\end{itemize}

O mapa, portanto, funciona como uma interface navegável para os dados gerados pelo \textit{pipeline}. Ele apresenta a organização dos tópicos e sua distribuição espacial, permitindo ao usuário explorar o \textit{dataset} e as relações temáticas de forma visual.

\section[Limitações do Estudo]{Limitações do Estudo}
\label{sec:limitacoes}

A validade e a generalização dos resultados apresentados neste capítulo devem ser consideradas no contexto de limitações metodológicas específicas, inerentes à natureza deste estudo de caso como um experimento de \gls{dsr}.

\begin{itemize}
    \item \textbf{Conjunto de Dados (Prototipagem vs. Produção):} A principal limitação deste estudo é a fonte dos dados. O \textit{pipeline} foi desenvolvido e validado sobre um \textit{dataset} de teste estático (um \textit{dump} do NPAI) e não sobre o acervo de dados completo e dinâmico do \textbf{Observatório de dados públicos de ciência e tecnologia da Bahia}. Embora este \textit{dataset} tenha sido suficiente para a validação do protótipo (o artefato), os resultados aqui apresentados (como a quantidade e a natureza dos tópicos) não são representativos do acervo real do Observatório.

    \item \textbf{Dependência dos Hiperparâmetros:} A configuração do \textit{pipeline}, especialmente os hiperparâmetros de modelagem (Seções \ref{sssec:reducao_dimensionalidade} e \ref{sssec:clusterizacao}), foi diretamente influenciada pelo volume reduzido do \textit{dataset} de teste. Parâmetros como \texttt{n\_neighbors=5} (UMAP) e \texttt{min\_cluster\_size=4} (HDBSCAN) foram ajustados para permitir a descoberta de tópicos de nicho em um \textit{corpus} pequeno. Estes parâmetros não seriam diretamente transferíveis para o acervo completo do Observatório, que, sendo ordens de magnitude maior, exigiria uma nova etapa de \textit{tuning} (ajuste fino).

    \item \textbf{Natureza do Texto de Entrada:} O \textit{pipeline} foi treinado utilizando exclusivamente os \textbf{títulos} das publicações. Textos curtos, por definição, oferecem baixa co-ocorrência de palavras. Este fator contextual é a explicação técnica mais provável para o escore negativo da métrica \gls{npmi} (Seção \ref{ssec:coerencia_topicos}), que é uma métrica estatística dependente de co-ocorrência. Os resultados de coerência poderiam ser diferentes se o modelo fosse treinado em textos mais longos, como os resumos (\textit{abstracts}).

    \item \textbf{Limitação da Métrica de Coerência:} Conforme discutido na Seção \ref{ssec:coerencia_topicos}, a própria métrica \gls{npmi} apresenta uma limitação contextual. Ela foi projetada para avaliar modelos estatísticos (como o \gls{lda}) e pode não ser a ferramenta de avaliação quantitativa ideal para modelos baseados em similaridade semântica (como o \gls{bertopic}). A validação qualitativa (Seção \ref{sec:validacao_qualitativa}) foi, portanto, necessária para avaliar a interpretabilidade dos tópicos.

    \item \textbf{Natureza do Artefato (DSR):} O artefato desenvolvido é um \textbf{protótipo} executado em um ambiente de desenvolvimento (Google Colab). Ele não está integrado ao \textit{pipeline} de produção, ao banco de dados dinâmico ou à interface de usuário existente do Observatório (descritos na Seção \ref{ssec:base_observatorio}). A sua função neste estudo é demonstrar a \textit{viabilidade} da metodologia, e não apresentar uma solução de \textit{software} em produção.
\end{itemize}