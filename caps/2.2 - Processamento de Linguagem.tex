\section{Processamento de Linguagem Natural (PLN)}

O Processamento de Linguagem Natural (\gls{pln}) é um campo multidisciplinar, situado na interseção da Inteligência Artificial, da Linguística Computacional e da Ciência da Informação. O objetivo central da área é desenvolver métodos computacionais capazes de processar, analisar, compreender e gerar a linguagem humana — seja em formato de texto ou voz — de maneira funcional \cite{Jurafsky_2009}\footnote{Refere-se à obra \textit{Speech and Language Processing}, de Daniel Jurafsky e James H. Martin. É amplamente considerado o livro-texto acadêmico padrão e a referência canônica para o ensino e estudo do Processamento de Linguagem Natural em todo o mundo.}.

Historicamente, o \gls{pln} fundamentava-se em abordagens estatísticas e regras linguísticas manuais para modelar a linguagem, conforme descrito por \cite{Manning_1999}\footnote{Refere-se à obra \textit{Foundations of Statistical Natural Language Processing} (Manning e Schütze, 1999), considerada o trabalho seminal que consolidou as abordagens estatísticas como o padrão do \gls{pln} antes da ascensão das redes neurais profundas.}. As técnicas de \gls{pln} são projetadas para extrair significado e estrutura de dados textuais, que são inerentemente não estruturados. Este processo envolve uma série de tarefas que variam desde a análise sintática (a estrutura gramatical) até a análise semântica (o significado por trás das palavras). Entre as aplicações comuns estão a classificação de textos, a tradução automática, a sumarização de documentos e a Modelagem de Tópicos (\textit{Topic Modeling}), que é o foco desta pesquisa.

A evolução recente do campo foi impulsionada pelo \textit{Deep Learning} (Aprendizado Profundo), que permitiu a criação de representações vetoriais mais precisas. Como destacam \citeonline{Galli_2024}, o \gls{pln} moderno passou a depender da capacidade de capturar o significado contextual, superando a análise baseada apenas na contagem de palavras. Essa transição para uma abordagem focada na compreensão semântica viabilizou os avanços recentes em modelagem de tópicos.