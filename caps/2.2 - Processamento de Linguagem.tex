\section{Processamento de Linguagem Natural (PLN)}

O Processamento de Linguagem Natural (\gls{pln}) é um campo multidisciplinar, situado na interseção da Inteligência Artificial, da Linguística Computacional e da Ciência da Informação. O seu objetivo central é desenvolver métodos computacionais capazes de processar, analisar, compreender e gerar a linguagem humana — seja em formato de texto ou voz — de maneira útil e análoga à humana \cite{Jurafsky_2009}\footnote{Refere-se à obra \textit{Speech and Language Processing}, de Daniel Jurafsky e James H. Martin. É amplamente considerado o livro-texto acadêmico padrão e a referência canônica para o ensino e estudo do Processamento de Linguagem Natural em todo o mundo.}.

Historicamente, o \gls{pln} dependia de abordagens estatísticas e regras linguísticas manuais para modelar a linguagem \cite{Manning_1999}\footnote{Refere-se à obra \textit{Foundations of Statistical Natural Language Processing} (Manning e Schütze, 1999), considerada o trabalho seminal que consolidou as abordagens estatísticas como o padrão do \gls{pln} antes da ascensão das redes neurais profundas.}. As técnicas de \gls{pln} são projetadas para extrair significado e estrutura de dados textuais, que são inerentemente não estruturados. Isso envolve uma série de tarefas complexas, desde a análise sintática (a estrutura gramatical) até a análise semântica (o significado por trás das palavras). Tarefas comuns incluem a classificação de textos, a tradução automática, a sumarização de documentos e, de relevância particular para este referencial, a Modelagem de Tópicos (\textit{Topic Modeling}).

A evolução recente do campo foi impulsionada pelo \textit{Deep Learning} (Aprendizado Profundo), que permitiu a criação de representações vetoriais de alta qualidade. Como destaca \citeonline{Galli_2024}, o \gls{pln} moderno depende fundamentalmente da capacidade de capturar o significado contextual.

\begin{citacao}
Um componente essencial para alcançar a compreensão semântica são os \textit{embeddings} — representações numéricas que codificam o significado de palavras ou mesmo de frases — que são fundamentais no \gls{pln} para capturar relacionamentos complexos entre palavras e frases usando arquiteturas especiais conhecidas como transformadores. \citeonline[Traduzido, p. 2]{Galli_2024}
\end{citacao}

Dessa forma, o \gls{pln} evoluiu de uma análise baseada em contagem de palavras para uma abordagem focada na compreensão semântica, o que viabilizou os avanços em modelagem de tópicos discutidos nas seções seguintes.